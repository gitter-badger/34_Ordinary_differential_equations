\documentclass[12pt]{article}
\usepackage{pmmeta}
\pmcanonicalname{OrderNConstantCoefficientDifferentialEquationsAndMatrixExponential}
\pmcreated{2013-03-22 19:01:00}
\pmmodified{2013-03-22 19:01:00}
\pmowner{gaillard}{1824}
\pmmodifier{gaillard}{1824}
\pmtitle{order n constant coefficient differential equations and matrix exponential}
\pmrecord{7}{41887}
\pmprivacy{1}
\pmauthor{gaillard}{1824}
\pmtype{Definition}
\pmcomment{trigger rebuild}
\pmclassification{msc}{34-01}

% this is the default PlanetMath preamble.  as your knowledge
% of TeX increases, you will probably want to edit this, but
% it should be fine as is for beginners.

% almost certainly you want these
\usepackage{amssymb}
\usepackage{amsmath}
\usepackage{amsfonts}

% used for TeXing text within eps files
%\usepackage{psfrag}
% need this for including graphics (\includegraphics)
%\usepackage{graphicx}
% for neatly defining theorems and propositions
%\usepackage{amsthm}
% making logically defined graphics
%%%\usepackage{xypic}

% there are many more packages, add them here as you need them

% define commands here

\begin{document}
Let $P$ be a degree $n>0$ monic complex polynomial in one indeterminate, let $f$ be a continuous function on the real line, let $k$ be an integer varying from 0 to $n-1$, and let $y_k$ be a complex number. The solution to the ODE
%
\begin{equation}\label{se}P(d/d t)\ y=f(t),\quad
y^{(k)}(0)=y_k
\end{equation}
%
is
%
\begin{equation}\label{sse}
y(t)=\sum\ y_k\ g_k(t)+\int_0^t g_{n-1}(t-x)\ f(x)\ d x,
\end{equation}
%
where $g_k(t)$ is the coefficient of $z^k$ in the product of $P(z)$ by the singular part of
%
$$\frac{e^{t z}}{P(z)}\quad.$$
%
Moreover, if $A$ is a complex square matrix annihilated by $P$, then 
%
\begin{equation}\label{e}e^{t A}=\sum\ g_k(t)\ A^k.
\end{equation}

%%%%%%%%%%%%%%%%%%%%%%%%%%%%%%%%%%%%%%%%%%%%%%%%%%%%%%%%

\PMlinkescapetext{Transform} (\ref{se}) into
%
\begin{equation}\label{ve}
Y'-B\,Y=f(t)\ v,\quad Y(0)=Y_0
\end{equation}
%
by putting $Y_k:=y^{(k)}$, $Y_{0k}:=y_k$, and by letting $B$ be the transpose companion matrix of $P$, and $v$ the last vector of the canonical basis of $\mathbb{C}^n$. The solution to (\ref{ve}) is
%
$$Y(t)=e^{t B}\ Y_0+\int_0^t\ f(x)\ e^{(t-x)B}\ v\ d x.$$
%
There is a unique $n$-tuple of functions $h_k$ such that
$e^{t A}$ is the sum of the $h_k(t)\,A^k$ whenever $A$ is a complex square matrix annihilated by $P$. The first line of $B^k$ being the $(k+1)$-th vector of the canonical basis of $\mathbb{C}^n$ (for $0\le k<n$), we obtain
%
$$y(t)=\sum\ y_k\ h_k(t)
+\int_0^t h_{n-1}(t-x)\ f(x)\ d x,$$
%
so that the proof of (\ref{sse}) and (\ref{e}) boils down to verifying
%
$$h_k(t)=g_k(t).$$
%
\PMlinkescapetext{Fix} a real value of $t$, let $G\in\mathbb{C}[X]$ be the sum of the $g_k(t)\, X^k$, form the entire function 
%
$$\varphi(z)=\frac{e^{t z}-G(z)}{P(z)}\quad,$$
%
multiply the above equality by $P(z)$, and replace $z$ by $A$. 
%%%%%
%%%%%
\end{document}
