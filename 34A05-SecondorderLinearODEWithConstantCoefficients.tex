\documentclass[12pt]{article}
\usepackage{pmmeta}
\pmcanonicalname{SecondorderLinearODEWithConstantCoefficients}
\pmcreated{2014-03-01 17:02:54}
\pmmodified{2014-03-01 17:02:54}
\pmowner{pahio}{2872}
\pmmodifier{pahio}{2872}
\pmtitle{second-order linear ODE with constant coefficients}
\pmrecord{8}{88038}
\pmprivacy{1}
\pmauthor{pahio}{2872}
\pmtype{Derivation}
\pmclassification{msc}{34A05}

% this is the default PlanetMath preamble.  as your knowledge
% of TeX increases, you will probably want to edit this, but
% it should be fine as is for beginners.

% almost certainly you want these
\usepackage{amssymb}
\usepackage{amsmath}
\usepackage{amsfonts}

% need this for including graphics (\includegraphics)
\usepackage{graphicx}
% for neatly defining theorems and propositions
\usepackage{amsthm}

% making logically defined graphics
%\usepackage{xypic}
% used for TeXing text within eps files
%\usepackage{psfrag}

% there are many more packages, add them here as you need them

% define commands here

\begin{document}
Let's consider the ordinary second-order linear differential equation
\begin{align}
      \frac{d^2y}{dx^2}+a\frac{dy}{dx}+by \;=\; 0
\end{align}
which is 
\PMlinkname{homogeneous}{HomogeneousLinearDifferentialEquation} 
and the coefficients $a, b$ of which are constants.\, As 
mentionned in the entry 
``finding another particular solution of linear ODE'', a simple substitution 
makes possible to eliminate from it the addend containing first 
derivative of the unknown function.\, Therefore we 
concentrate upon the case\, $a = 0$.\, We have two cases 
depending on the sign of\, $b = \pm k^2$.\\

\textbf{$1^\circ$}.\; $b > 0$.\;  We will solve the equation
\begin{align}
\frac{d^2y}{dx^2}+k^2y \;=\; 0.
\end{align}
Multiplicating both addends by the expression $2\frac{dy}{dx}$ it becomes
$$2\frac{dy}{dx}\frac{d^2y}{dx^2}+2k^2y\frac{dy}{dx} \;=\; 0,$$
where the left hand side is the derivative of 
$\left(\frac{dy}{dx}\right)^2+k^2y^2$.\, The latter one thus has a constant value 
which must be nonnegative; denote it by $k^2C^2$.\, We then have the equation
\begin{align}
\left(\frac{dy}{dx}\right)^2 \;=\; k^2(C^2-y^2).
\end{align}
After taking the square root and separating the variables it reads
$$\frac{dy}{\pm\sqrt{C^2\!-y^2}} \;=\; k\,dx.$$
Integrating (see the table of integrals) this yields
$$\arcsin\frac{y}{C} \;=\; k(x\!-\!x_0)$$
where $x_0$ is another constant.\, Consequently, the general 
solution of the differential equation (2) may be written
\begin{align}
y \,\;=\; C\,\sin k(x\!-\!x_0)
\end{align}
in which $C$ and $x_0$ are arbitrary real constants. 

If one denotes\, $C\cos kx_0 = C_1$\, and $-C\sin kx_0 = C_2$,
then (4) reads
\begin{align}
y \,\;=\; C_1\sin kx+C_2\cos kx.
\end{align}
Here, $C_1$ and $C_2$ are arbitrary constants.\, Because both 
$\sin kx$ and $\cos kx$  satisfy the given equation (2) and are 
linearly independent, its general solution can be written as (5).\\

\textbf{$2^\circ$}.\; $b < 0$.\;  An analogical treatment of 
the equation
\begin{align}
      \frac{d^2y}{dx^2}-k^2y \;=\; 0.
\end{align}
yields for it the general solution
\begin{align}
y \,\;=\; C_1e^{kx}+C_2e^{-kx}
\end{align}
(note that one can eliminate the square root from the equation
$y\pm\sqrt{y^2+C} = C'e^{kx}$ and its ``inverted equation'' 
$y\mp\sqrt{y^2+C} = -\frac{C}{C'}e^{-kx}$).\, The linear independence 
of the obvious solutions $e^{\pm kx}$ implies also the linear 
independence of $\cosh kx$ and $\sinh kx$ and thus allows us to
give the general solution also in the alternative form
\begin{align}
y \,\;=\; C_1\sinh kx+C_2\cosh kx.
\end{align}\\

\textbf{Remark.}\, The standard method for solving a 
\PMlinkname{homogeneous}{HomogeneousLinearDifferentialEquation}
ordinary second-order linear differential equation (1) with 
constant coefficients is to use in it the 
substitution
\begin{align}
y \;=\; e^{rx}
\end{align}
where $r$ is a constant; see the entry ``second order linear 
differential equation with constant coefficients''.\, This method 
is possible to use also for such equations of higher order.\\



\begin{thebibliography}{8}
\bibitem{lindelof}{\sc Ernst Lindel\"of}: {\em Differentiali- ja integralilasku
ja sen sovellutukset III.1}.\, Mercatorin Kirjapaino Osakeyhti\"o, Helsinki (1935).
\end{thebibliography}\\



\end{document}
