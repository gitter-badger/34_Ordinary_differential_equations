\documentclass[12pt]{article}
\usepackage{pmmeta}
\pmcanonicalname{FuchsianSingularity}
\pmcreated{2013-03-22 14:47:26}
\pmmodified{2013-03-22 14:47:26}
\pmowner{rspuzio}{6075}
\pmmodifier{rspuzio}{6075}
\pmtitle{Fuchsian singularity}
\pmrecord{14}{36442}
\pmprivacy{1}
\pmauthor{rspuzio}{6075}
\pmtype{Definition}
\pmcomment{trigger rebuild}
\pmclassification{msc}{34A25}
\pmsynonym{Fuchsian singular point}{FuchsianSingularity}
\pmsynonym{regular singular point}{FuchsianSingularity}
\pmsynonym{regular singularity}{FuchsianSingularity}
\pmrelated{FrobeniusMethod}
\pmdefines{irregular singular point}
\pmdefines{irregular singularity}
\pmdefines{Hamburger equation}

% this is the default PlanetMath preamble.  as your knowledge
% of TeX increases, you will probably want to edit this, but
% it should be fine as is for beginners.

% almost certainly you want these
\usepackage{amssymb}
\usepackage{amsmath}
\usepackage{amsfonts}

% used for TeXing text within eps files
%\usepackage{psfrag}
% need this for including graphics (\includegraphics)
%\usepackage{graphicx}
% for neatly defining theorems and propositions
%\usepackage{amsthm}
% making logically defined graphics
%%%\usepackage{xypic}

% there are many more packages, add them here as you need them

% define commands here
\begin{document}
Suppose that $D$ is an open subset of $\mathbb{C}$ and that the $n$ functions $c_k 
\colon D \to \mathbb{C}, \quad k = 0, \ldots, n-1$ are meromorphic.  Consider the ordinary differential equation
 $${d^n w \over dz^n} + \sum_{k=0}^{n-1} c_k (z) {d^k w \over dz^k} = 0$$
A point $p \in D$ is said to be a \emph{regular singular point} or a \emph{Fuchsian singular point} of this equation if at least one of the functions $c_k$ has a pole at $p$ and, for every value of $k$ between $0$ and $n$, either $c_k$ is regular at $p$ or has a pole of order not greater than $n-k$.  

If $p$ is a Fuchsian singular point, then the differential equation may be rewritten as a system of first order equations
 $${d v_i \over dz} = {1 \over z} \sum_{j=1}^{n} b_{ij}(z) v_j (z)$$
in which the coefficient functions $b_{ij}$ are analytic at $z$.  This fact helps explain the restiction on the orders of the poles of the $c_k$'s.

If an equation has a Fuchsian singularity, then the solution can be expressed as a Frobenius series in a neighborhood of this point.

A singular point of a differential equation which is not a regular singular point is known as an irregular singular point.

{\bf Examples}

The Bessel equation
 $$w'' +  {1 \over z} w' + {z^2 - 1 \over z^2} w = 0$$
has a Fuchsian singularity at $z=0$ since the coefficient of $w'$ has a pole of order $1$ and the coefficient of $w$ has a pole of order $2$.

On the other hand, the \emph{Hamburger equation}
 $$w'' + {2 \over z} w' + {z^2 - 1 \over z^4} w = 0$$
has an irregular singularity at $z=0$ since the coefficient of $w$ has a pole of order $4$.
%%%%%
%%%%%
\end{document}
