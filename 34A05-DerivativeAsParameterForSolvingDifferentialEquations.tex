\documentclass[12pt]{article}
\usepackage{pmmeta}
\pmcanonicalname{DerivativeAsParameterForSolvingDifferentialEquations}
\pmcreated{2013-03-22 18:28:39}
\pmmodified{2013-03-22 18:28:39}
\pmowner{pahio}{2872}
\pmmodifier{pahio}{2872}
\pmtitle{derivative as parameter for solving differential equations}
\pmrecord{10}{41150}
\pmprivacy{1}
\pmauthor{pahio}{2872}
\pmtype{Topic}
\pmcomment{trigger rebuild}
\pmclassification{msc}{34A05}
%\pmkeywords{ordinary differential equation}
\pmrelated{InverseFunctionTheorem}
\pmrelated{ImplicitFunctionTheorem}
\pmrelated{DAlembertsEquation}
\pmrelated{ClairautsEquation}

% this is the default PlanetMath preamble.  as your knowledge
% of TeX increases, you will probably want to edit this, but
% it should be fine as is for beginners.

% almost certainly you want these
\usepackage{amssymb}
\usepackage{amsmath}
\usepackage{amsfonts}

% used for TeXing text within eps files
%\usepackage{psfrag}
% need this for including graphics (\includegraphics)
%\usepackage{graphicx}
% for neatly defining theorems and propositions
 \usepackage{amsthm}
% making logically defined graphics
%%%\usepackage{xypic}

% there are many more packages, add them here as you need them

% define commands here

\theoremstyle{definition}
\newtheorem*{thmplain}{Theorem}

\begin{document}
\PMlinkescapeword{independent}

The solution of some differential equations of the forms \,$x = f(\frac{dy}{dx})$\, and\, $y = f(\frac{dy}{dx})$\, may be  expressed in a parametric form by taking for the parameter the derivative
\begin{align}
p \;:=\; \frac{dy}{dx}.
\end{align}

\textbf{I.}\; Consider first the equation
\begin{align}
x = f(\frac{dy}{dx}),
\end{align}
for which we suppose that\, $p \mapsto f(p)$\, and its derivative\, $p \mapsto f'(p)$\, are continuous and\, 
$f'(p) \neq 0$\, on an interval \,$[p_1,\,p_2]$.\, It follows that on the interval, the function \,$p \mapsto f(p)$\, changes monotonically from\, $f(p_1) := x_1$\, to\, $f(p_2) := x_2$, whence conversely the equation
\begin{align}
x = f(p)
\end{align}
defines from\, $[p_1,\,p_2]$\, onto\, $[x_1,\,x_2]$\, a bijection
\begin{align}
p = g(x)
\end{align}
which is continuously differentiable.\, Thus on the interval\, $[x_1,\,x_2]$,\, the differential equation (2) can be replaced by the equation
\begin{align}
\frac{dy}{dx} = g(x),
\end{align}
and therefore, the solution of (2) is
\begin{align}
y = \int g(x)\,dx+C.
\end{align}
If we cannot express $g(x)$ in a \PMlinkescapetext{closed form}, we take $p$ as an independent variable through the substitution (3), which maps \,$[x_1,\,x_2]$\, bijectively onto\, $[p_1,\,p_2]$.\, Then (6) becomes a function of $p$, and by the chain rule,
$$\frac{dy}{dp} = g(f(p))f'(p) = pf'(p).$$
Accordingly, the solution of the given differential equation may be presented on\, $[p_1,\,p_2]$\, as

\begin{align}
\begin{cases}
\displaystyle x = f(p),\\
\displaystyle y = \int\!p\,f'(p)\,dp+C.
\end{cases}
\end{align}


\textbf{II.}\; With corresponding considerations, one can write the solution of the differential equation
\begin{align}
y = f(\frac{dy}{dx}) \;:=\; f(p),
\end{align}
where $p$ changes on some interval\, $[p_1,\,p_2]$\, where $f(p)$ and $f'(p)$ are continuous and\, 
$p\cdot f'(p) \neq 0$,\, in the parametric presentation
\begin{align}
\begin{cases}
\displaystyle x = \int\!\frac{f'(p)}{p}\,dp+C,\\
\displaystyle y = f(p).
\end{cases}
\end{align}


\textbf{III.}\; The procedures of \textbf{I} and \textbf{II} may be generalised for the differential equations of 
\PMlinkescapetext{types} \,$x = f(y,\,p)$\, and\, $y = f(x,\,p)$;\, let's consider the former one.

In
\begin{align}
x \;=\; f(y,\,p)
\end{align}
we regard $y$ as the independent variable and differentiate with respect to it:
$$\frac{dx}{dy} \;=\; \frac{1}{p} \;=\; f'_y(y,\,p)\!+\!f'_p(y,\,p)\frac{dp}{dy}.$$
Supposing that the partial derivative \,$f'_p(y,\,p)$ does not vanish identically, we get
\begin{align}
 \frac{dp}{dy} \;=\; \frac{\frac{1}{p}-f'_y(y,\,p)}{f'_p(y,\,p)} \;:=\; g(y,\,p).
\end{align}
If\, $p = p(y,\,C)$\, is the general solution of (11), we obtain the general solution of (10):
\begin{align}
x \;=\; f(y,\,p(y,C))
\end{align}


%%%%%
%%%%%
\end{document}
