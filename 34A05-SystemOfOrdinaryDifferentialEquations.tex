\documentclass[12pt]{article}
\usepackage{pmmeta}
\pmcanonicalname{SystemOfOrdinaryDifferentialEquations}
\pmcreated{2014-03-07 16:37:51}
\pmmodified{2014-03-07 16:37:51}
\pmowner{pahio}{2872}
\pmmodifier{pahio}{2872}
\pmtitle{system of ordinary differential equations}
\pmrecord{8}{88048}
\pmprivacy{1}
\pmauthor{pahio}{2872}
\pmtype{Topic}
\pmclassification{msc}{34A05}

\endmetadata

% this is the default PlanetMath preamble.  as your knowledge
% of TeX increases, you will probably want to edit this, but
% it should be fine as is for beginners.

% almost certainly you want these
\usepackage{amssymb}
\usepackage{amsmath}
\usepackage{amsfonts}

% need this for including graphics (\includegraphics)
\usepackage{graphicx}
% for neatly defining theorems and propositions
\usepackage{amsthm}

% making logically defined graphics
%\usepackage{xypic}
% used for TeXing text within eps files
%\usepackage{psfrag}

% there are many more packages, add them here as you need them

% define commands here

\begin{document}
In many problems one would have to find the functions 
$y_1(x)$, $y_2(x)$, $\ldots$, $y_n(x)$ satisfying the differential equation 
system of the form
\begin{align}
\begin{cases}
  \frac{dy_1}{dx} \;=\; f_1(x,y_1,y_2,\ldots,y_n),\\
  \frac{dy_2}{dx} \;=\; f_2(x,y_1,y_2,\ldots,y_n),\\
   \ldots\qquad\ldots\qquad\ldots\\
  \frac{dy_n}{dx} \;=\; f_n(x,y_1,y_2,\ldots,y_n).             
\end{cases}
\end{align}
Such a system is called {\it normal system.}\, The functions $f_i$ are supposed 
to be differentiable.\, Usually there are also 
given some initial conditions
\begin{align}
y_1(x_0) = y_{01}, \quad y_2(x_0) = y_{02}, \;\;\ldots,\; y_n(x_0) = y_{0n}.
\end{align}

The solving procedure may be as follows.

First we differentiate the first equation (1) with respect to the argument $x$:
$$\frac{d^2y_1}{dx^2} \;=\; \frac{\partial f_1}{\partial x}
+\frac{\partial f_1}{\partial y_1}\frac{dy_1}{dx}+\ldots
+\frac{\partial f_1}{\partial y_n}\frac{dy_n}{dx}$$
Here one substitutes the derivatives $\frac{dy_i}{dx}$ as they 
are given by the equations (1), getting the equation of the 
form
$$\frac{d^2y_1}{dx^2} \;=\; F_2(x,y_1,\ldots,y_n).$$ 
When one differentiates this equation and makes the substitutions 
as above, the result has the form
$$\frac{d^3y_1}{dx^3} \;=\; F_3(x,y_1,\ldots,y_n).$$ 
Then one can continue similarly and will finally come to the 
system
\begin{align}
\begin{cases}
  \frac{dy_1}{dx} \;=\; f_1(x,y_1,y_2,\ldots,y_n),\\
  \frac{d^2y_1}{dx^2} \;=\; F_2(x,y_1,y_2,\ldots,y_n),\\
   \ldots\qquad\ldots\qquad\ldots\\
  \frac{d^ny_1}{dx^n} \;=\; F_n(x,y_1,y_2,\ldots,y_n).             
\end{cases}
\end{align}
The $n\!-\!1$ first equations (3) determine $y_2,y_3,\ldots,y_n$ 
as functions of $x$, $y_1$, $y_1',\ldots,y_1^{(n-1)}$:
\begin{align}
\begin{cases}
  y_2 \;=\; \varphi_2(x,y_1,y_1',\ldots,y_1^{(n-1)}),\\
  y_3 \;=\; \varphi_3(x,y_1,y_1',\ldots,y_1^{(n-1)}),\\
   \ldots\qquad\ldots\qquad\ldots\\
  y_n \;=\; \varphi_n(x,y_1,y_1',\ldots,y_1^{(n-1)}).             
\end{cases}
\end{align}
These expressions of $y_2,y_3,\ldots,y_n$ are put into the last 
of the equations (3), and then one has an $n$'th order 
differential equation for solving $y_1$:
\begin{align}
  \frac{d^ny_1}{dx^n} \;=\; \Phi(x,y_1,y_1',\ldots,y_1^{(n-1)}).
\end{align}
Solving this gives the function
\begin{align}
  y_1 \;=\; \psi(x,C_1,C_2,\ldots,C_n)}.
\end{align}
Differentiating this $n\!-\!1$ times yields the derivatives 
$\frac{dy_1}{dx},\frac{d^2y_1}{dx^2},\ldots,
\frac{d^{n-1}y_1}{dx^{n-1}}$ as functions of 
$x,C_1,C_2,\ldots,C_n$.\, These derivatives are put into the 
equations (4), giving the functions $y_2,y_3,\ldots,y_n$:
\begin{align}
\begin{cases}
  y_2 \;=\; \psi_2(x,C_1,C_2,\ldots,C_n),\\
   \ldots\qquad\ldots\qquad\ldots\\
  y_n \;=\; \psi_n(x,C_1,C_2,\ldots,C_n).            
\end{cases}
\end{align}
In the solution (6) and (7), one has still to consider the initial 
conditions (2); then the constants $C_i$ of integration attain 
certain values.\\

\textbf{Remark.}\, If the system (1) is linear, then also the 
equation (5) is linear.\\


\textbf{Example.}\, Solve the functions $y(x)$ and $z(x)$ from the pair 
of differential equations
\begin{align}
\begin{cases}
  \frac{dy}{dx} \;=\; \,x\!+\!y\!+\!z,\\
  \frac{dz}{dx} \;=\; 2x\!-\!4y\!-\!3z
\end{cases}
\end{align}
subject to the initial conditions\, $y(0) = 1$\, and\, 
$z(0) =0.$

Differentiation of the first equation with respect to $x$ gives
$$\frac{d^2y}{dx^2} \;=\; 1+\frac{dy}{dx}+\frac{dz}{dx}.$$
Setting to this the first derivatives from (8) turns it into
\begin{align}
 \frac{d^2y}{dx^2} \;=\; 3x-3y-2z+1. 
\end{align}
Into this we put the expression
\begin{align}
 z \;=\; \frac{dy}{dx}\!-\!x\!-\!y
\end{align}
 got 
from the first equation (8), obtaing the second order linear 
differential equation
$$\frac{d^2y}{dx^2}+2\frac{dy}{dx}+y \;=\; 5x\!+\!1$$
with constant coefficients.
The general solution of this last equation is
\begin{align}
  y = (C_1+C_2x)e^{-x}+5x-9, 
\end{align}
and by (10) this yields
\begin{align}
  z \;=\;  (C_2-2C_1-2C_2x)e^{-x}-6x-14. 
\end{align}
The initial conditions give from (11) and (12)
$$C_1-9 \;=\; 1, \qquad C_2-2C_1+14 \;=\; 0,$$
whence\, $C_1 = 10$\, and\, $C_2 = 6$\, and thus the 
particular solution in question is
\begin{align*}
\begin{cases}
  y \;=\; (6x+10)e^{-x}+5x-9,\\
  z \;=\; -(12x+14)e^{-x}-6x+14.
\end{cases}
\end{align*}




\begin{thebibliography}{9}
\bibitem{NP}{\sc N. Piskunov:} {\em Diferentsiaal- ja 
integraalarvutus k\~{o}rgematele tehnilistele 
\~{o}ppeasutustele. Teine k\"{o}ide.} Viies tr\"{u}kk.\, 
Kirjastus Valgus, Tallinn  (1966).
\end{thebibliography}\\




\end{document}
