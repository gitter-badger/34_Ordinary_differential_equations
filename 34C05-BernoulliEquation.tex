\documentclass[12pt]{article}
\usepackage{pmmeta}
\pmcanonicalname{BernoulliEquation}
\pmcreated{2013-03-22 15:15:03}
\pmmodified{2013-03-22 15:15:03}
\pmowner{pahio}{2872}
\pmmodifier{pahio}{2872}
\pmtitle{Bernoulli equation}
\pmrecord{11}{37032}
\pmprivacy{1}
\pmauthor{pahio}{2872}
\pmtype{Derivation}
\pmcomment{trigger rebuild}
\pmclassification{msc}{34C05}
\pmsynonym{Bernoulli differential equation}{BernoulliEquation}
\pmrelated{RiccatiEquation}

\endmetadata

% this is the default PlanetMath preamble.  as your knowledge
% of TeX increases, you will probably want to edit this, but
% it should be fine as is for beginners.

% almost certainly you want these
\usepackage{amssymb}
\usepackage{amsmath}
\usepackage{amsfonts}

% used for TeXing text within eps files
%\usepackage{psfrag}
% need this for including graphics (\includegraphics)
%\usepackage{graphicx}
% for neatly defining theorems and propositions
 \usepackage{amsthm}
% making logically defined graphics
%%%\usepackage{xypic}

% there are many more packages, add them here as you need them

% define commands here

\theoremstyle{definition}
\newtheorem*{thmplain}{Theorem}
\begin{document}
The {\em Bernoulli equation} has the form
\begin{align}
\frac{dy}{dx}+f(x)y \;=\; g(x)y^k
\end{align}
where $f$ and $g$ are continuous real functions and $k$ is a \PMlinkescapetext{constant} ($\neq 0$, \,$\neq 1$).\, Such a \PMlinkname{nonlinear equation}{DifferentialEquation} is got e.g. in examining the motion of a \PMlinkescapetext{body when the resistance of medium depends on the velocity $v$ as
$$F \;=\; \lambda_1v\!+\!\lambda_2v^k.$$
The real function $y$ can be solved from (1) explicitly.\, To do this, divide first both sides} by $y^k$.\, It yields
\begin{align}
y^{-k}\frac{dy}{dx}+f(x)y^{-k+1} \;=\; g(x).
\end{align}
The substitution
\begin{align}
z \;:=\; y^{-k+1}
\end{align}
transforms (2) into 
$$\frac{dz}{dx}+(-k\!+\!1)f(x)z \;=\; (-k\!+\!1)g(x)$$
which is a linear differential equation of first order.\, When one has obtained its general solution and made in this the substitution (3), then one has solved the Bernoulli equation (1).

\begin{thebibliography}{9}
\bibitem{NP}{\sc N. Piskunov:} {\em Diferentsiaal- ja integraalarvutus k\~{o}rgematele tehnilistele \~{o}ppeasutustele}. \,-- Kirjastus Valgus, Tallinn  (1966).
\end{thebibliography}
%%%%%
%%%%%
\end{document}
