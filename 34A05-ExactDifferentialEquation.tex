\documentclass[12pt]{article}
\usepackage{pmmeta}
\pmcanonicalname{ExactDifferentialEquation}
\pmcreated{2013-03-22 18:06:17}
\pmmodified{2013-03-22 18:06:17}
\pmowner{pahio}{2872}
\pmmodifier{pahio}{2872}
\pmtitle{exact differential equation}
\pmrecord{10}{40648}
\pmprivacy{1}
\pmauthor{pahio}{2872}
\pmtype{Result}
\pmcomment{trigger rebuild}
\pmclassification{msc}{34A05}
\pmrelated{HarmonicConjugateFunction}
\pmrelated{Differential}
\pmrelated{TotalDifferential}
\pmdefines{exact differential equation}

\endmetadata

% this is the default PlanetMath preamble.  as your knowledge
% of TeX increases, you will probably want to edit this, but
% it should be fine as is for beginners.

% almost certainly you want these
\usepackage{amssymb}
\usepackage{amsmath}
\usepackage{amsfonts}

% used for TeXing text within eps files
%\usepackage{psfrag}
% need this for including graphics (\includegraphics)
%\usepackage{graphicx}
% for neatly defining theorems and propositions
 \usepackage{amsthm}
% making logically defined graphics
%%%\usepackage{xypic}

% there are many more packages, add them here as you need them

% define commands here

\theoremstyle{definition}
\newtheorem*{thmplain}{Theorem}

\begin{document}
Let $R$ be a region in $\mathbb{R}^2$ and let the functions\, $X\!: R \to \mathbb{R}$,\, $Y\!: R \to \mathbb{R}$ have continuous partial derivatives in $R$.\, The first order differential equation
$$X(x,\,y)+Y(x,\,y)\frac{dy}{dx} \;=\; 0$$
or
\begin{align}
X(x,\,y)dx+Y(x,\,y)dy \;=\; 0
\end{align}
is called an {\em exact differential equation}, if the condition
\begin{align}
\frac{\partial X}{\partial y} \;=\; \frac{\partial Y}{\partial x}
\end{align}
is true in $R$.

By (2), the left hand side of (1) is the total differential of a function, there is a function\, $f\!: R \to \mathbb{R}$\, such that the equation (1) reads
$$d\,f(x,\,y) \;=\; 0,$$
whence its general integral is
$$f(x,\,y) \;=\; C.$$

The solution function $f$ can be calculated as the line integral
\begin{align}
f(x,\,y) \;:=\; \int_{P_0}^P [X(x,\,y)\,dx+Y(x,\,y)\,dy]
\end{align}
along any curve $\gamma$ connecting an arbitrarily chosen point \,$P_0 = (x_0,\,y_0)$\, and the point\, $P = (x,\,y)$\, in the region $R$ (the integrating factor is now $\equiv 1$).\\

\textbf{Example.}\, Solve the differential equation
$$\frac{2x}{y^3}\,dx+\frac{y^2-3x^2}{y^4}\,dy \;=\; 0.$$
This equation is exact, since
$$\frac{\partial}{\partial y}\frac{2x}{y^3} \;=\; -\frac{6x}{y^4} 
\;=\; \frac{\partial}{\partial x}\frac{y^2-3x^2}{y^4}.$$
If we use as the integrating way the broken line from\, $(0,\,1)$\, to\, $(x,\,1)$\, and from this to\, $(x,\,y)$,\, the integral (3) is simply
$$\int_0^x\frac{2x}{1^3}\,dx+\!\int_1^y\frac{y^2-3x^2}{y^4}\,dy \;=\; \frac{x^2}{y^3}-\frac{1}{y}+1 
\;=\; x^2-\frac{1}{y}+\frac{x^2}{y^3}+1-x^2 = \frac{x^2}{y^3}-\frac{1}{y}+1.$$
Thus we have the general integral
$$\frac{x^2}{y^3}-\frac{1}{y} \;=\; C$$
of the given differential equation.
%%%%%
%%%%%
\end{document}
