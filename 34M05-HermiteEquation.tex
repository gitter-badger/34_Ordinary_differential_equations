\documentclass[12pt]{article}
\usepackage{pmmeta}
\pmcanonicalname{HermiteEquation}
\pmcreated{2013-03-22 15:16:15}
\pmmodified{2013-03-22 15:16:15}
\pmowner{pahio}{2872}
\pmmodifier{pahio}{2872}
\pmtitle{Hermite equation}
\pmrecord{19}{37058}
\pmprivacy{1}
\pmauthor{pahio}{2872}
\pmtype{Definition}
\pmcomment{trigger rebuild}
\pmclassification{msc}{34M05}
\pmsynonym{Hermite differential equation}{HermiteEquation}
\pmrelated{ChebyshevEquation}

% this is the default PlanetMath preamble.  as your knowledge
% of TeX increases, you will probably want to edit this, but
% it should be fine as is for beginners.

% almost certainly you want these
\usepackage{amssymb}
\usepackage{amsmath}
\usepackage{amsfonts}

% used for TeXing text within eps files
%\usepackage{psfrag}
% need this for including graphics (\includegraphics)
%\usepackage{graphicx}
% for neatly defining theorems and propositions
 \usepackage{amsthm}
% making logically defined graphics
%%%\usepackage{xypic}

% there are many more packages, add them here as you need them

% define commands here

\theoremstyle{definition}
\newtheorem*{thmplain}{Theorem}
\begin{document}
The linear differential equation 
        $$\frac{d^2f}{dz^2}-2z\frac{df}{dz}+2nf \;=\; 0,$$
in which $n$ is a real \PMlinkescapetext{constant}, is called the {\em Hermite equation}.\, Its general solution is\, $f := Af_1\!+\!Bf_2$\, with $A$ and $B$ arbitrary \PMlinkescapetext{constants} and the functions $f_1$ and $f_2$ presented as\\

\quad $f_1(z) \;:=\; z+\frac{2(1-n)}{3!}z^3+\frac{2^2(1-n)(3-n)}{5!}z^5+
\frac{2^3(1-n)(3-n)(5-n)}{7!}z^7+\ldots\!,$\\

\quad $f_2(z) \;:=\; 1+\frac{2(-n)}{2!}z^2+\frac{2^2(-n)(2-n)}{4!}z^4+
\frac{2^3(-n)(2-n)(4-n)}{6!}z^6+\ldots$\\

It's easy to check that these power series satisfy the differential equation.\, The coefficients $b_\nu$ in both series obey the recurrence \PMlinkescapetext{formula}
    $$b_\nu \;=\; \frac{2(\nu\!-\!2\!-\!n)}{\nu(nu\!-\!1)}b_{\nu\!-\!2}.$$
Thus we have the \PMlinkname{radii of convergence}{RadiusOfConvergence}
 $$R \;=\; \lim_{\nu\to\infty}\left|\frac{b_{\nu-2}}{b_\nu}\right| \;=\; 
\lim_{\nu\to\infty}\frac{\nu}{2}\!\cdot\!\frac{1\!-\!1/\nu}{1\!-\!(n\!+\!2)/\nu} \;=\; \infty.$$
Therefore the series converge in the whole complex plane and define entire functions.

If the \PMlinkescapetext{constant} $n$ is a non-negative integer, then one of $f_1$ and $f_2$ is simply a polynomial function.\, The polynomial solutions of the Hermite equation are usually normed so that the highest \PMlinkname{degree}{PolynomialRing} \PMlinkescapetext{term} is $(2z)^n$ and called the Hermite polynomials.
%%%%%
%%%%%
\end{document}
