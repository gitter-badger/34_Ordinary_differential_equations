\documentclass[12pt]{article}
\usepackage{pmmeta}
\pmcanonicalname{LinearDifferentialEquationOfFirstOrder}
\pmcreated{2013-03-22 16:32:09}
\pmmodified{2013-03-22 16:32:09}
\pmowner{pahio}{2872}
\pmmodifier{pahio}{2872}
\pmtitle{linear differential equation of first order}
\pmrecord{13}{38717}
\pmprivacy{1}
\pmauthor{pahio}{2872}
\pmtype{Derivation}
\pmcomment{trigger rebuild}
\pmclassification{msc}{34A30}
\pmsynonym{linear ordinary differential equation of first order}{LinearDifferentialEquationOfFirstOrder}
\pmrelated{SeparationOfVariables}

\endmetadata

% this is the default PlanetMath preamble.  as your knowledge
% of TeX increases, you will probably want to edit this, but
% it should be fine as is for beginners.

% almost certainly you want these
\usepackage{amssymb}
\usepackage{amsmath}
\usepackage{amsfonts}

% used for TeXing text within eps files
%\usepackage{psfrag}
% need this for including graphics (\includegraphics)
%\usepackage{graphicx}
% for neatly defining theorems and propositions
 \usepackage{amsthm}
% making logically defined graphics
%%%\usepackage{xypic}

% there are many more packages, add them here as you need them

% define commands here

\theoremstyle{definition}
\newtheorem*{thmplain}{Theorem}

\begin{document}
An ordinary linear differential equation of first order has the form
\begin{align}
  \frac{dy}{dx}+P(x)y \;=\; Q(x),
\end{align}
where $y$ means the unknown function, $P$ and $Q$ are two known continuous functions.\\

For finding the solution of (1), we may seek a function $y$ which is product of two functions:
\begin{align}
  y(x) \;=\; u(x)v(x)
\end{align}
One of these two can be chosen freely; the other is determined according to (1).

We substitute (2) and the derivative \,$\frac{dy}{dx} = u\frac{dv}{dx}+v\frac{du}{dx}$\, in (1), getting\, 
$u\frac{dv}{dx}+v\frac{du}{dx}+Puv = Q$,\, or
\begin{align}
  u\left(\frac{dv}{dx}+Pv\right)+v\frac{du}{dx} \;=\; Q.
\end{align}
If we chose the function $v$ such that
$$\frac{dv}{dx}+Pv \;=\; 0,$$
this condition may be written
$$\frac{dv}{v} \;=\; -P\,dx.$$
Integrating here both sides gives\, $\ln{v} = -\int P\,dx$\, or
$$v \;=\; e^{-\int Pdx},$$
where the exponent means an arbitrary antiderivative of\, $-P$.\, Naturally, $v(x) \neq 0$.\,

Considering the chosen property of $v$ in (3), this equation can be written
$$v\frac{du}{dx} \;=\; Q,$$
i.e. 
$$\frac{du}{dx} \;=\; \frac{Q(x)}{v(x)},$$
whence
$$u \;=\; \int\frac{Q(x)}{v(x)}\,dx+C \;=\; C+\!\int Qe^{\int Pdx}dx.$$

So we have obtained the solution
\begin{align}
  y \;=\; e^{-\int P(x)dx}\left[C+\!\int Q(x)e^{\int P(x)dx}dx\right]
\end{align}
of the given differential equation (1).

The result (4) presents the general solution of (1), since the arbitrary \PMlinkescapetext{constant} $C$ may be always chosen so that any given initial condition
$$y \;=\; y_0 \quad \mathrm{when}\quad x \;=\; x_0$$
is fulfilled.

%%%%%
%%%%%
\end{document}
