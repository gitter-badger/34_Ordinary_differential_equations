\documentclass[12pt]{article}
\usepackage{pmmeta}
\pmcanonicalname{MaximalIntervalOfExistenceOfOrdinaryDifferentialEquations}
\pmcreated{2013-03-22 13:37:06}
\pmmodified{2013-03-22 13:37:06}
\pmowner{Daume}{40}
\pmmodifier{Daume}{40}
\pmtitle{maximal interval of existence of ordinary differential equations}
\pmrecord{8}{34251}
\pmprivacy{1}
\pmauthor{Daume}{40}
\pmtype{Theorem}
\pmcomment{trigger rebuild}
\pmclassification{msc}{34A12}
\pmclassification{msc}{35-00}
\pmclassification{msc}{34-00}

% this is the default PlanetMath preamble.  as your knowledge
% of TeX increases, you will probably want to edit this, but
% it should be fine as is for beginners.

% almost certainly you want these
\usepackage{amssymb}
\usepackage{amsmath}
\usepackage{amsfonts}

% used for TeXing text within eps files
%\usepackage{psfrag}
% need this for including graphics (\includegraphics)
%\usepackage{graphicx}
% for neatly defining theorems and propositions
%\usepackage{amsthm}
% making logically defined graphics
%%%\usepackage{xypic} 

% there are many more packages, add them here as you need them

% define commands here
\begin{document}
Let $E\subset W$ where $W$ is a normed vector space, $f\in C^1(E)$ is a continuous differentiable map $f: E \to W$.  Furthermore consider the ordinary differential equation
$$\dot{x} = f(x)$$
with the initial condition
\begin{center}
$x(0) = x_0$.
\end{center}
For all $x_0 \in E$ there exists a unique solution
$$x:I \to E$$
where $I = [-a,a]$, which also satify the initial condition of the initial value problem.  Then there exists a maximal interval of existence $J=( \alpha ,\beta )$ such that $I\subset J$ and there exists a unique solution   
\begin{center}
$x:J \to E$.
\end{center}
%%%%%
%%%%%
\end{document}
