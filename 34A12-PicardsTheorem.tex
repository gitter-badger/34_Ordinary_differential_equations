\documentclass[12pt]{article}
\usepackage{pmmeta}
\pmcanonicalname{PicardsTheorem}
\pmcreated{2013-03-22 14:59:57}
\pmmodified{2013-03-22 14:59:57}
\pmowner{Daume}{40}
\pmmodifier{Daume}{40}
\pmtitle{Picard's theorem}
\pmrecord{6}{36706}
\pmprivacy{1}
\pmauthor{Daume}{40}
\pmtype{Theorem}
\pmcomment{trigger rebuild}
\pmclassification{msc}{34A12}
\pmsynonym{Picard-Lindel\"of theorem}{PicardsTheorem}
\pmrelated{ExistenceAndUniquenessOfSolutionOfOrdinaryDifferentialEquations}

\endmetadata

% this is the default PlanetMath preamble.  as your knowledge
% of TeX increases, you will probably want to edit this, but
% it should be fine as is for beginners.

% almost certainly you want these
\usepackage{amssymb}
\usepackage{amsmath}
\usepackage{amsfonts}
\usepackage{amsthm}

% used for TeXing text within eps files
%\usepackage{psfrag}
% need this for including graphics (\includegraphics)
%\usepackage{graphicx}
% making logically defined graphics
%%%\usepackage{xypic} 

% there are many more packages, add them here as you need them

% define commands here

% The below lines should work as the command
% \renewcommand{\bibname}{References}
% without creating havoc when rendering an entry in
% the page-image mode.
\makeatletter
\@ifundefined{bibname}{}{\renewcommand{\bibname}{References}}
\makeatother

\newtheorem{thm}{Theorem}
\newtheorem{defn}{Definition}
\newtheorem{prop}{Proposition}
\newtheorem{lemma}{Lemma}
\newtheorem{cor}{Corollary}
\begin{document}
\begin{thm}[Picard's theorem \cite{KF}]
Let $E$ be an open subset of $\mathbb{R}^2$ and a continuous function $f(x,y)$ defined as $f\colon E \to \mathbb{R}$.  If $(x_0,y_0)\in E$ and $f$ satisfies the Lipschitz condition in the variable $y$ in $E$:
$$|f(x,y)-f(x,y_1)| \leq M| y-y_1|$$
where $M$ is a constant.  Then the ordinary differential equation defined as
$$\frac{dy}{dx} = f(x,y)$$
with the initial condition
$$y(x_0) = y_0$$
has a unique solution $y(x)$ on some interval $|x- x_0| \leq\delta$.
\end{thm}

The above theorem is also named the \emph{Picard-Lindel\"of theorem} and can be generalized to a system of first order ordinary differential equations

\begin{thm}[generalization of Picard's theorem \cite{KF}]
Let $E$ be an open subset of $\mathbb{R}^{n+1}$ and a continuous function $f(x,y_1,\ldots,y_n)$ defined as $f=(f_1,\ldots,f_n)\colon E \to \mathbb{R}^n$.  If $(t_0,y_{10},\ldots,y_{n0})\in E$ and $f$ satisfies the Lipschitz condition in the variable $y_1,\ldots,y_n$ in $E$:
$$|f_i(x,y_1,\dots,y_n)-f_i(x,y_1'\ldots,y_n')| \leq M \max_{1\leq j\leq n}| y_j-y_j'|$$
where $M$ is a constant.  Then the system of ordinary differential equation defined as
\begin{align*}
\frac{dy_1}{dx} & =  f_1(x,y_1,\ldots,y_n)\\
& \vdots \\
\frac{dy_n}{dx} & =  f_n(x,y_1,\ldots,y_n)
\end{align*}
with the initial condition
$$y_1(x_0) = y_{10},\ldots, y_n(x_0) = y_{n0}$$
has a unique solution
$$y_1(x) ,\ldots, y_n(x)$$
on some interval $|x- x_0| \leq\delta$.
\end{thm}

\textbf{see also:}
\begin{itemize}
\item the entry existence and uniqueness of solution of ordinary differential equations
\end{itemize}

\begin{thebibliography}{1}
\bibitem[KF]{KF} Kolmogorov, A.N. \& Fomin, S.V.: Introductory Real Analysis, Translated \& Edited by Richard A. Silverman. Dover Publications, Inc. New York, 1970.
\end{thebibliography}
%%%%%
%%%%%
\end{document}
