\documentclass[12pt]{article}
\usepackage{pmmeta}
\pmcanonicalname{ContactPoint}
\pmcreated{2013-03-22 14:06:22}
\pmmodified{2013-03-22 14:06:22}
\pmowner{Mathprof}{13753}
\pmmodifier{Mathprof}{13753}
\pmtitle{contact point}
\pmrecord{6}{35507}
\pmprivacy{1}
\pmauthor{Mathprof}{13753}
\pmtype{Definition}
\pmcomment{trigger rebuild}
\pmclassification{msc}{34-00}
\pmclassification{msc}{34A99}

\endmetadata

% this is the default PlanetMath preamble.  as your knowledge
% of TeX increases, you will probably want to edit this, but
% it should be fine as is for beginners.

% almost certainly you want these
\usepackage{amssymb}
\usepackage{amsmath}
\usepackage{amsfonts}

% used for TeXing text within eps files
%\usepackage{psfrag}
% need this for including graphics (\includegraphics)
%\usepackage{graphicx}
% for neatly defining theorems and propositions
%\usepackage{amsthm}
% making logically defined graphics
%%%\usepackage{xypic} 

% there are many more packages, add them here as you need them

% define commands here
\begin{document}
\PMlinkescapeword{planar}
Let $\phi$ be a planar dynamical system and $\gamma$ a curve.  Then a \emph{contact point} of the curve and the dynamical system occurs when the tangent vector of the curve is collinear to the vector of the dynamical system.
\cite{1}
\begin{thebibliography}{1}
\bibitem[KGA]{1} Khovanski\u\i, A.G.: Fewnomials. American Mathematical Society, Providence, 1991.
\end{thebibliography}
%%%%%
%%%%%
\end{document}
