\documentclass[12pt]{article}
\usepackage{pmmeta}
\pmcanonicalname{BesselsEquation}
\pmcreated{2013-03-22 16:34:57}
\pmmodified{2013-03-22 16:34:57}
\pmowner{pahio}{2872}
\pmmodifier{pahio}{2872}
\pmtitle{Bessel's equation}
\pmrecord{26}{38774}
\pmprivacy{1}
\pmauthor{pahio}{2872}
\pmtype{Definition}
\pmcomment{trigger rebuild}
\pmclassification{msc}{34A05}
\pmclassification{msc}{33C10}
\pmsynonym{Bessel's differential equation}{BesselsEquation}
\pmsynonym{Bessel equation}{BesselsEquation}
\pmrelated{LaplaceEquationInCylindricalCoordinates}
\pmrelated{CauchyResidueTheorem}
\pmrelated{PropertiesOfEntireFunctions}
\pmrelated{FrobeniusMethod}
\pmrelated{TableOfLaplaceTransforms}
\pmrelated{BesselFunctionsAndHelicalStructureDiffractionPatterns}
\pmdefines{Bessel's function}
\pmdefines{Bessel function}

% this is the default PlanetMath preamble.  as your knowledge
% of TeX increases, you will probably want to edit this, but
% it should be fine as is for beginners.

% almost certainly you want these

\usepackage{amssymb}
\usepackage{amsmath}
\usepackage{amsfonts}

% used for TeXing text within eps files
%\usepackage{psfrag}
% need this for including graphics (\includegraphics)
%\usepackage{graphicx}
% for neatly defining theorems and propositions
 \usepackage{amsthm}
% making logically defined graphics
%%%\usepackage{xypic}

% there are many more packages, add them here as you need them

% define commands here


\begin{document}
\PMlinkescapeword{constant} \PMlinkescapeword{order}
The linear differential equation 
\begin{align}
  x^2\frac{d^2y}{dx^2}+x\frac{dy}{dx}+(x^2-p^2)y \;=\; 0,
\end{align}
in which $p$ is a constant (non-negative if it is real), is called the {\em Bessel's equation}.\, We derive its general solution by trying the series form
\begin{align}
               y \;=\; x^r\sum_{k=0}^\infty a_kx^k \;=\; \sum_{k=0}^\infty a_kx^{r+k},
\end{align}
due to Frobenius.\, Since the parameter $r$ is indefinite, we may regard $a_0$ as distinct from 0.

We substitute (2) and the derivatives of the series in (1):
$$
x^2\sum_{k=0}^\infty(r+k)(r+k-1)a_kx^{r+k-2}+
  x\sum_{k=0}^\infty(r+k)a_kx^{r+k-1}+
(x^2-p^2)\sum_{k=0}^\infty a_kx^{r+k} \;=\; 0.
$$
Thus the coefficients of the powers $x^r$, $x^{r+1}$, $x^{r+2}$ and so on must vanish, and we get the system of equations
\begin{align}
\begin{cases}
{[}r^2-p^2{]}a_0 \;=\; 0,\\
{[}(r+1)^2-p^2{]}a_1 \;=\; 0,\\
{[}(r+2)^2-p^2{]}a_2+a_0 \;=\; 0,\\
\qquad \qquad \ldots\\
{[}(r+k)^2-p^2{]}a_k+a_{k-2} \;=\; 0.
\end{cases}
\end{align}
The last of those can be written
$$(r+k-p)(r+k+p)a_k+a_{k-2} \;=\; 0.$$
Because\, $a_0 \neq 0$,\, the first of those (the indicial equation) gives\, $r^2-p^2 = 0$,\, i.e. we have the roots
$$r_1 \;=\; p, \quad r_2 \;=\; -p.$$
Let's first look the the solution of (1) with\, $r = p$;\, then\, $k(2p+k)a_k+a_{k-2} = 0$,\, and thus\,
$$a_k \;=\; -\frac{a_{k-2}}{k(2p+k).}$$
From the system (3) we can solve one by one each of the coefficients $a_1$, $a_2$, $\ldots$\, and express them with $a_0$ which remains arbitrary.\, Setting for $k$ the integer values we get
\begin{align}
\begin{cases}
a_1 \;=\; 0, \quad a_3 \;=\; 0,\;\ldots,\; a_{2m-1} \;=\; 0;\\
a_2 \;=\; -\frac{a_0}{2(2p+2)}, \quad a_4 \;=\; \frac{a_0}{2\cdot4(2p+2)(2p+4)},\;\ldots,\;\,
a_{2m} \;=\; \frac{(-1)^ma_0}{2\cdot4\cdot6\cdots(2m)(2p+2)(2p+4)\ldots(2p+2m)}
\end{cases}
\end{align}
(where\, $m = 1,\,2,\,\ldots$).
Putting the obtained coefficients to (2) we get the particular solution 
\begin{align}
 y_1 \;:=\; a_0x^p \left[\!\frac{x^2}{2(2p\!+\!2)}\!
+\!\frac{x^4}{2\!\cdot\!4(2p\!+\!2)(2p\!+\!4)}
\!-\!\frac{x^6}{2\!\cdot\!4\!\cdot\!6(2p\!+\!2)(2p\!+\!4)(2p\!+\!6)}\!+-\ldots\right]
\end{align}

In order to get the coefficients $a_k$ for the second root\, $r_2 = -p$\, we have to look after that
$$(r_2+k)^2-p^2 \;\neq\; 0,$$
or\, $r_2+k \neq p = r_1$.\, Therefore
$$r_1-r_2 \;=\; 2p \;\neq\; k$$
where $k$ is a positive integer.\, Thus, when $p$ is not an integer and not an integer added by $\frac{1}{2}$, we get the second particular solution, gotten of (5) by replacing $p$ by $-p$:
\begin{align}
 y_2 \;:=\; a_0x^{-p}\!\left[1
\!-\!\frac{x^2}{2(-2p\!+\!2)}\!+\!\frac{x^4}{2\!\cdot\!4(-2p\!+\!2)(-2p\!+\!4)}
\!-\!\frac{x^6}{2\!\cdot\!4\!\cdot\!6(-2p\!+\!2)(-2p\!+\!4)(-2p\!+\!6)}\!+-\ldots\right]
\end{align}

The power series of (5) and (6) converge for all values of $x$ and are linearly independent (the ratio $y_1/y_2$ tends to 0 as\, $x\to\infty$).\, With the appointed value
$$a_0 \;=\; \frac{1}{2^p\,\Gamma(p+1)},$$
the solution $y_1$ is called the {\em Bessel function of the first kind and of order $p$} and denoted by $J_p$.\, The similar definition is set for the first kind Bessel function of an arbitrary order\, $p\in \mathbb{R}$ (and $\mathbb{C}$).
 For\, $p\notin \mathbb{Z}$\, the general solution of the Bessel's differential equation is thus
$$y \;:=\; C_1J_p(x)+C_2J_{-p}(x),$$
where\, $J_{-p}(x) = y_2$\, with\, $a_0 = \frac{1}{2^{-p}\Gamma(-p+1)}$.

The explicit expressions for $J_{\pm p}$ are
\begin{align}
J_{\pm p}(x) \;=\; 
 \sum_{m=0}^\infty 
  \frac{(-1)^m}{m!\,\Gamma(m\pm p+1)}\left(\frac{x}{2}\right)^{2m\pm p},
\end{align}
which are obtained from (5) and (6) by using the last \PMlinkescapetext{formula} for gamma function.

E.g. when\, $p = \frac{1}{2}$\, the series in (5) gets the form
$$y_1 \;=\; \frac{x^{\frac{1}{2}}}{\sqrt{2}\,\Gamma(\frac{3}{2})}\left[1\!-\!\frac{x^2}{2\!\cdot\!3}\!+\!\frac{x^4}{2\!\cdot\!4\!\cdot\!3\!\cdot\!5}\!-\!\frac{x^6}{2\!\cdot\!4\cdot\!6\!\cdot\!3\!\cdot\!5\!\cdot\!7}\!+-\ldots\right] \;=\;
\sqrt{\frac{2}{\pi x}}\left(x\!-\!\frac{x^3}{3!}\!+\!\frac{x^5}{5!}\!-+\ldots\right).$$
Thus we get
$$J_{\frac{1}{2}}(x) \;=\; \sqrt{\frac{2}{\pi x}}\sin{x};$$
analogically (6) yields
$$J_{-\frac{1}{2}}(x) \;=\; \sqrt{\frac{2}{\pi x}}\cos{x},$$
and the general solution of the equation (1) for\, $p = \frac{1}{2}$\, is
$$y \;:=\; C_1J_{\frac{1}{2}}(x)+C_2J_{-\frac{1}{2}}(x).$$


In the case that $p$ is a non-negative integer $n$, the ``+'' case of (7) gives the solution
$$J_{n}(x) \;=\; 
 \sum_{m=0}^\infty 
  \frac{(-1)^m}{m!\,(m+n)!}\left(\frac{x}{2}\right)^{2m+n},
$$
but for\, $p = -n$\, the expression of $J_{-n}(x)$ is $(-1)^nJ_n(x)$, i.e. linearly dependent on $J_n(x)$.\, It can be shown that the other solution of (1) ought to be searched in the form\, 
$y = K_n(x) = J_n(x)\ln{x}+x^{-n}\sum_{k=0}^\infty b_kx^k$.\, Then the general solution is\, $y := C_1J_n(x)+C_2K_n(x)$.\\

\textbf{Other formulae}

The first kind Bessel functions of integer order have the generating function $F$:
\begin{align}
F(z,\,t) \;=\; e^{\frac{z}{2}(t-\frac{1}{t})} \;=\; \sum_{n=-\infty}^\infty J_n(z)t^n
\end{align}
This function has an essential singularity at\, $t = 0$\, but is analytic elsewhere in $\mathbb{C}$; thus $F$ has the Laurent expansion in that point.\, Let us prove (8) by using the general expression
$$c_n \;=\; \frac{1}{2\pi i}\oint_{\gamma} \frac{f(t)}{(t-a)^{n+1}}\,dt$$
of the coefficients of Laurent series.\, Setting to this\, $a := 0$,\, 
$f(t) := e^{\frac{z}{2}(t-\frac{1}{t})}$,\, $\zeta := \frac{zt}{2}$\, gives
$$c_n \;=\; \frac{1}{2\pi i}
\oint_\gamma\frac{e^{\frac{zt}{2}}e^{-\frac{z}{2t}}}{t^{n+1}}\,dt \;=\; 
\frac{1}{2\pi i}\left(\frac{z}{2}\right)^n\!
\oint_\delta\frac{e^\zeta e^{-\frac{z^2}{4\zeta}}}{\zeta^{n+1}}\,d\zeta \;=\; 
\sum_{m=0}^\infty\frac{(-1)^m}{m!}\left(\frac{z}{2}\right)^{2m+n}\!
\frac{1}{2\pi i}\oint_\delta \zeta^{-m-n-1}e^\zeta\,d\zeta.$$
The paths $\gamma$ and $\delta$ go once round the origin anticlockwise in the $t$-plane and $\zeta$-plane, respectively.\, Since the residue of $\zeta^{-m-n-1}e^\zeta$ in the origin is\, $\frac{1}{(m+n)!} = \frac{1}{\Gamma(m+n+1)}$,\, the \PMlinkname{residue theorem}{CauchyResidueTheorem} gives
$$c_n \;=\; \sum_{m=0}^\infty
\frac{(-1)^m}{m!\Gamma(m+n+1)}\left(\frac{z}{2}\right)^{2m+n} \;=\; J_n(z).$$
This \PMlinkescapetext{means} that $F$ has the Laurent expansion (8).

By using the generating function, one can easily derive other formulae, e.g.
the \PMlinkescapetext{integral representation} of the Bessel functions of integer order:
$$J_n(z) \;=\; \frac{1}{\pi}\int_0^\pi\cos(n\varphi-z\sin{\varphi})\,d\varphi$$
Also one can obtain the addition formula
$$J_n(x\!+\!y) \;=\; \sum_{\nu=-\infty}^{\infty}J_\nu(x)J_{n-\nu}(y)$$
and the series \PMlinkescapetext{representations} of cosine and sine:
$$\cos{z} \;=\; J_0(z)-2J_2(z)+2J_4(z)-+\ldots$$
$$\sin{z} \;=\; 2J_1(z)-2J_3(z)+2J_5(z)-+\ldots$$



\begin{thebibliography}{9}
\bibitem{NP}{\sc N. Piskunov:} {\em Diferentsiaal- ja integraalarvutus k\~{o}rgematele tehnilistele \~{o}ppeasutustele}.\, Kirjastus Valgus, Tallinn  (1966).
\bibitem{KK}{\sc K. Kurki-Suonio:} {\em Matemaattiset apuneuvot}.\, Limes r.y., Helsinki (1966).
\end{thebibliography}


%%%%%
%%%%%
\end{document}
