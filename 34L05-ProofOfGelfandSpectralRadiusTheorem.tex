\documentclass[12pt]{article}
\usepackage{pmmeta}
\pmcanonicalname{ProofOfGelfandSpectralRadiusTheorem}
\pmcreated{2013-03-22 15:33:55}
\pmmodified{2013-03-22 15:33:55}
\pmowner{Andrea Ambrosio}{7332}
\pmmodifier{Andrea Ambrosio}{7332}
\pmtitle{proof of Gelfand spectral radius theorem}
\pmrecord{7}{37469}
\pmprivacy{1}
\pmauthor{Andrea Ambrosio}{7332}
\pmtype{Proof}
\pmcomment{trigger rebuild}
\pmclassification{msc}{34L05}

\endmetadata

% this is the default PlanetMath preamble.  as your knowledge
% of TeX increases, you will probably want to edit this, but
% it should be fine as is for beginners.

% almost certainly you want these
\usepackage{amssymb}
\usepackage{amsmath}
\usepackage{amsfonts}

% used for TeXing text within eps files
%\usepackage{psfrag}
% need this for including graphics (\includegraphics)
%\usepackage{graphicx}
% for neatly defining theorems and propositions
%\usepackage{amsthm}
% making logically defined graphics
%%%\usepackage{xypic}

% there are many more packages, add them here as you need them

% define commands here
\begin{document}
For any $\epsilon>0$, consider the matrix 
\[
\tilde{A}=(\rho(A)+\epsilon)^{-1}A
\]
Then, obviously,
\[
\rho(\tilde{A}) = \frac{\rho(A)}{\rho(A)+\epsilon}<1
\]
and, by a well-known result on convergence of matrix powers,
\[
\lim_{k \to \infty}\tilde{A}^k=0.
\]
That means, by sequence limit definition, a natural number $N_1\in \mathbf{N}$ exists such that
\[
\forall k\geq N_1 \Rightarrow \|\tilde{A}^k\| < 1
\]
which in turn means:
\[
\forall k\geq N_1 \Rightarrow \|A^k\| < (\rho(A)+\epsilon)^k
\]
or 
\[
\forall k\geq N_1 \Rightarrow \|A^k\|^{1/k} < (\rho(A)+\epsilon).
\]

Let's now consider the matrix 
\[
\check{A}=(\rho(A)-\epsilon)^{-1}A
\]
Then, obviously,
\[
\rho(\check{A}) = \frac{\rho(A)}{\rho(A)-\epsilon}>1
\]
and so, by the same convergence theorem,$\|\check{A}^k\|$ is not bounded.
This means a natural number $N_2\in \mathbf{N}$ exists such that
\[
\forall k\geq N_2 \Rightarrow \|\check{A}^k\|>1
\]
which in turn means:
\[
\forall k\geq N_2 \Rightarrow \|A^k\| > (\rho(A)-\epsilon)^k
\]
or 
\[
\forall k\geq N_2 \Rightarrow \|A^k\|^{1/k} > (\rho(A)-\epsilon).
\]
Taking $N:=max(N_1,N_2)$ and putting it all together, we obtain:
\[
\forall \epsilon>0, \exists N\in\mathbb{N}: \forall k\geq N \Rightarrow \rho(A)-\epsilon < \|A^k\|^{1/k} < \rho(A)+\epsilon
\]
which, by definition, is
\[
\lim_{k \to \infty}\|A^k\|^{1/k} = \rho(A).\,\,\square
\]

Actually, in case the norm is \PMlinkname{self-consistent}{SelfConsistentMatrixNorm}, the proof shows more than the thesis; in fact, using the fact that $|\lambda|\leq\rho(A)$, we can replace in the limit definition the left lower bound with the spectral radius itself and write more precisely:
\[
\forall \epsilon>0, \exists N\in\mathbb{N}: \forall k\geq N \Rightarrow \rho(A) \leq \|A^k\|^{1/k} < \rho(A)+\epsilon
\]
which, by definition, is
\[\lim_{k \to \infty}\|A^k\|^{1/k} = \rho(A)^+.
\]
%%%%%
%%%%%
\end{document}
