\documentclass[12pt]{article}
\usepackage{pmmeta}
\pmcanonicalname{OrthogonalCurve}
\pmcreated{2013-03-22 14:50:10}
\pmmodified{2013-03-22 14:50:10}
\pmowner{pahio}{2872}
\pmmodifier{pahio}{2872}
\pmtitle{orthogonal curve}
\pmrecord{17}{36504}
\pmprivacy{1}
\pmauthor{pahio}{2872}
\pmtype{Definition}
\pmcomment{trigger rebuild}
\pmclassification{msc}{34C99}
\pmclassification{msc}{34C05}
\pmsynonym{orthogonal trajectory}{OrthogonalCurve}
%\pmkeywords{family of curves}
\pmrelated{ConditionOfOrthogonality}
\pmrelated{HarmonicConjugateFunction}
\pmrelated{ConvexAngle}
\pmrelated{Isocline}
\pmrelated{TiltCurve}
\pmrelated{HyperbolasOrthogonalToEllipses}
\pmrelated{IsogonalTrajectory}

\usepackage{amssymb}
\usepackage{amsmath}
\usepackage{amsfonts}
\usepackage{pstricks}
\usepackage{pst-plot}
\begin{document}
If a family of plane curves (with one free parameter) satisfies the differential equation
$$F(x,\,y,\,y') \;=\; 0,$$
where \,$y' = \frac{dy}{dx}$, then the family of curves \PMlinkname{intersecting orthogonally}{ConvexAngle} all the first curves satisfies the differential equation
$$F\left(x,\,y,\,-\frac{1}{y'}\right) \;=\; 0.$$
Anyone of the latter curves is an \emph{orthogonal curve} of the former ones.

\textbf{Example.} \,Let's consider the family of rectangular hyperbolas
$$x^2-y^2 \;=\; c$$
with the parameter $c$ taking any real value. \,Derivating with respect to $x$ gives the differential equation of this family,
$$x-yy' \;=\; 0,$$
and by replacing here $y'$ with $-\frac{1}{y'}$ we obtain the differential equation
$$x+\frac{y}{y'} \;=\; 0$$
of the orthogonal curves. \,Integrating its form
$$\frac{dy}{y} \;=\; -\frac{dx}{x}$$
gives the solution
$$xy \;=\; C,$$
which \PMlinkescapetext{represents} another family of rectangular hyperbolas.

 In the picture below (by drini), there are four hyperbolas of the first family (blue) given by the values \,$c = -1,\,-2,\,-4,\,-8$\, and four hyperbolas of the orthogonal family (red) given by the values \,$C = 1,\,2,\,4,\,8$.

\begin{center}
\begin{pspicture*}(-5.4,-5.4)(5.2,5.4)
\psaxes[labels=none,ticks=none](0,0)(-5,-5)(5,5)
\psgrid[subgriddiv=1,griddots=10,gridlabels=7pt](-5,-5)(5,5)
\psplot[linecolor=blue,linewidth=1pt]{-5}{5}{x x mul 1 add sqrt}
\psplot[linecolor=blue,linewidth=1pt]{-5}{5}{x x mul 2 add sqrt}
\psplot[linecolor=blue,linewidth=1pt]{-5}{5}{x x mul 4 add sqrt}
\psplot[linecolor=blue,linewidth=1pt]{-5}{5}{x x mul 8 add sqrt}
\psplot[linecolor=blue,linewidth=1pt]{-5}{5}{x x mul 1 add sqrt -1 mul}
\psplot[linecolor=blue,linewidth=1pt]{-5}{5}{x x mul 2 add sqrt -1 mul}
\psplot[linecolor=blue,linewidth=1pt]{-5}{5}{x x mul 4 add sqrt -1 mul}
\psplot[linecolor=blue,linewidth=1pt]{-5}{5}{x x mul 8 add sqrt -1 mul}
\psplot[linecolor=red,linewidth=1pt]{-5}{-0.1}{2 x div }
\psplot[linecolor=red,linewidth=1pt]{0.1}{5}{2 x div }
\psplot[linecolor=red,linewidth=1pt]{-5}{-0.1}{1 x div }
\psplot[linecolor=red,linewidth=1pt]{0.1}{5}{1 x div }
\psplot[linecolor=red,linewidth=1pt]{-5}{-0.1}{8 x div }
\psplot[linecolor=red,linewidth=1pt]{0.1}{5}{8 x div }
\psplot[linecolor=red,linewidth=1pt]{-5}{-0.1}{4 x div }
\psplot[linecolor=red,linewidth=1pt]{0.1}{5}{4 x div }
\end{pspicture*}
\end{center}
%%%%%
%%%%%
\end{document}
