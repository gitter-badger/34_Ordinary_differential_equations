\documentclass[12pt]{article}
\usepackage{pmmeta}
\pmcanonicalname{LiouvillesTheorem}
\pmcreated{2013-03-22 15:14:55}
\pmmodified{2013-03-22 15:14:55}
\pmowner{Koro}{127}
\pmmodifier{Koro}{127}
\pmtitle{Liouville's theorem}
\pmrecord{20}{37027}
\pmprivacy{1}
\pmauthor{Koro}{127}
\pmtype{Theorem}
\pmcomment{trigger rebuild}
\pmclassification{msc}{34A34}

\endmetadata

% this is the default PlanetMath preamble.  as your knowledge
% of TeX increases, you will probably want to edit this, but
% it should be fine as is for beginners.

% almost certainly you want these
\usepackage{amssymb}
\usepackage{amsmath}
\usepackage{amsfonts}
\usepackage{amsthm}

% used for TeXing text within eps files
%\usepackage{psfrag}
% need this for including graphics (\includegraphics)
%\usepackage{graphicx}
% making logically defined graphics
%%%\usepackage{xypic} 

% there are many more packages, add them here as you need them

% define commands here

% The below lines should work as the command
% \renewcommand{\bibname}{References}
% without creating havoc when rendering an entry in
% the page-image mode.
\makeatletter
\@ifundefined{bibname}{}{\renewcommand{\bibname}{References}}
\makeatother

\newtheorem{thm}{Theorem}
\newtheorem{defn}{Definition}
\newtheorem{prop}{Proposition}
\newtheorem{lemma}{Lemma}
\newtheorem{cor}{Corollary}
\begin{document}
Let
\begin{equation}
\dot{x}=f(x) \label{eq}
\end{equation}
be a autonomous ordinary differential equation in $\mathbb{R}^n$
defined by a smooth vector field $f\colon \mathbb{R}^n\to \mathbb{R}^n$ 
and the Jacobian of $f$ is denoted $\frac{\partial f}{\partial x}$. Also let $\Phi_t(x)$ be the \PMlinkname{flow}{Flow2} associated 
with (\ref{eq}).  Let 
$$V(t) = \int_{\Phi_t(D)} dx$$ 
be the volume of the image of $D$ under this flow after a time $t$.
\begin{thm}[Liouville's theorem]
If $D\subseteq \mathbb{R}^n$ is a bounded measurable domain. Then
$$\dot{V}(t)= \int_{\Phi_t(D)} \operatorname{div}\, f(x) dx$$
\end{thm}
\begin{proof}
Let $V(t)$ be defined as above then
\begin{eqnarray*}
V(t_0+h) 
& = & \int_{\Phi_{t_0+h}(D)}dy\\
& = & \int_{\Phi_h(\Phi_{t_0}(D))}dy\\
& = & \int_{\Phi_{t_0}(D)} \operatorname{det}\left(\frac{\partial\Phi_h}{\partial x}(x)\right) dx.
\end{eqnarray*}

We claim that, for $x\in \Phi_{t_0}(D)$,
$$\frac{\partial\Phi_t}{\partial x}(x) = I + t\frac{\partial f}{\partial x}(x) + o(t)$$
as $t\to 0$.

In fact, 
$$\Phi_t(x) = x + \int_{0}^t f(\Phi_s(x))ds,$$
and by the Leibniz integral rule
$$\frac{\partial \Phi_t}{\partial x}(x) = I + \int_{0}^t \frac{\partial}{\partial x}f(\Phi_s(x)) ds,$$
so that 
$$\frac{\partial}{\partial t} \frac{\partial \Phi_t}{\partial x}(x) = \frac{\partial}{\partial x}f(\Phi_t(x))$$
and evaluating at $t=0$ we get
$${\frac{\partial}{\partial t} \frac{\partial \Phi_t}{\partial x}(x)}\Big|_{t=0} = \frac{\partial}{\partial x}f(\Phi_0(x)) = \frac{\partial f}{\partial x} (x).$$
Our claim follows from this and from the definition of derivative.

Hence
\begin{eqnarray*}
\operatorname{det}\left(\frac{\partial\Phi_t}{\partial x}(x)\right) & = & \operatorname{det}\left(I + t\frac{\partial f}{\partial x}(x)\right) + o(t)\\
& = & \prod_{i=1}^n(1 + \frac{\partial f_i}{\partial x_i}(x)) + o(t)\\
& = & 1+t\sum_{i=1}^n\frac{\partial f_i}{\partial x_i}(x) +o(t)\\
& = & 1 + t\operatorname{div}\, f(x) + o(t)
\end{eqnarray*}
as $t\to0$.
It follows that 
$$V(t_0+h) = \int_{\Phi_{t_0}(D)} 1 + h\operatorname{div}\, f(x) + o(h) dx$$
and
\begin{eqnarray*}
\dot{V}(t_0) & = & \lim_{h\to 0}\frac{V(t_0+h)-V(t_0)}{h}\\
& = & \frac{\int_{\Phi_{t_0}(D)} 1 + h\operatorname{div}\, f(x) + o(h) dx -V(t_0)}{h}\\
& = & \frac{V(t_0) + h\int_{\Phi_{t_0}(D)}\operatorname{div}\, f(x)dx + o(h) -V(t_0)}{h}\\
& = & \int_{\Phi_{t_0}(D)}\operatorname{div}\, f(x)dx + \lim_{h\to 0}\frac{o(h)}{h}\\
& = & \int_{\Phi_{t_0}(D)}\operatorname{div}\, f(x)dx.
\end{eqnarray*}
\end{proof}

\begin{cor}
The flow of an \PMlinkname{Hamiltonian system}{HamiltonianEquations} preserves volume.
\end{cor}
\begin{proof}
It follows directly since the vector field of an Hamiltonian system has divergence equal to zero.  Hence $\dot{V}=0$ implies that the volume is constant.
\end{proof}

\begin{thebibliography}{1}
\bibitem[TG]{TG} Teschl, Gerald: Ordinary Differential Equations and Dynamical Systems. \PMlinkexternal{http://www.mat.univie.ac.at/~gerald/ftp/book-ode/index.html}{http://www.mat.univie.ac.at/~gerald/ftp/book-ode/index.html}, 2004.
\end{thebibliography}
%%%%%
%%%%%
\end{document}
