\documentclass[12pt]{article}
\usepackage{pmmeta}
\pmcanonicalname{ExampleOfSolvingAFunctionalEquation}
\pmcreated{2013-03-22 15:30:00}
\pmmodified{2013-03-22 15:30:00}
\pmowner{pahio}{2872}
\pmmodifier{pahio}{2872}
\pmtitle{example of solving a functional equation}
\pmrecord{14}{37359}
\pmprivacy{1}
\pmauthor{pahio}{2872}
\pmtype{Example}
\pmcomment{trigger rebuild}
\pmclassification{msc}{34A30}
\pmclassification{msc}{39B05}
\pmrelated{ChainRule}
\pmrelated{AdditionFormula}
\pmrelated{SubtractionFormula}
\pmrelated{DefinitionsInTrigonometry}
\pmrelated{GoniometricFormulae}
\pmrelated{DifferenceOfSquares}
\pmrelated{AdditionFormulas}

% this is the default PlanetMath preamble.  as your knowledge
% of TeX increases, you will probably want to edit this, but
% it should be fine as is for beginners.

% almost certainly you want these
\usepackage{amssymb}
\usepackage{amsmath}
\usepackage{amsfonts}

% used for TeXing text within eps files
%\usepackage{psfrag}
% need this for including graphics (\includegraphics)
%\usepackage{graphicx}
% for neatly defining theorems and propositions
 \usepackage{amsthm}
% making logically defined graphics
%%%\usepackage{xypic}

% there are many more packages, add them here as you need them

% define commands here

\theoremstyle{definition}
\newtheorem*{thmplain}{Theorem}
\begin{document}
Let's determine all twice differentiable real functions $f$ which satisfy the functional equation
\begin{align}
f(x\!+\!y) \cdot f(x\!-\!y) = [f(x)]^2\!-\![f(y)]^2
\end{align}
for all real values of $x$ and $y$.

Substituting first\, $y = 0$ in (1) we see that\, $f(x)^2 = f(x)^2\!-\!f(0)^2$\, or\, $f(0) = 0$.\, The substitution\, $x = 0$\, gives\, $f(y)f(-y) = -f(y)^2$,\, whence\, $f(-y) = -f(y)$.\, So $f$ is an odd function.

We differentiate both sides of (1) with respect to $y$ and the result with respect to $x$:
     $$f'(x\!+\!y)f(x\!-\!y)\!-\!f'(x\!-\!y)f(x\!+\!y) = -2f(y)f'(y)$$
 $$f''(x\!+\!y)f(x\!-\!y)\!+\!f'(x\!-\!y)f'(x\!+\!y)\!-\! f''(x\!-\!y)f(x\!+\!y)\!-\!f'(x\!+\!y)f'(x\!-\!y) = 0$$
The result is simplified to\, $f''(x\!+\!y)f(x\!-\!y) = f''(x\!-\!y)f(x\!+\!y)$,\, i.e.                      
              $$f''(x\!+\!y)/f(x\!+\!y) = f''(x\!-\!y)/f(x\!-\!y).$$
Denoting\, $x\!+\!y := u$,\, $x\!-\!y := v$\, we obtain the equation
                     $$\frac{f''(u)}{f(u)} = \frac{f''(v)}{f(v)}$$
for all real values of $u$ and $v$.\, This is not possible unless the proportion  $\frac{f''(u)}{f(u)}$ has a \PMlinkescapetext{constant value, independent} on $u$.\, Thus the \PMlinkescapetext{second order} homogeneous linear differential equation\, $f''(t)/f(t) = \pm{k}^2$ or
                   $$f''(t) = \pm{k}^2f(t),$$
with $k$ some \PMlinkescapetext{constant}, is valid.

There are three cases:
\begin{enumerate}
  \item $k = 0$.\, Now\, $f''(t) \equiv 0$\, and consequently\, $f(t) \equiv Ct$.\, If one especially \PMlinkescapetext{chooses the constant} $C$ equal to 1, the solution is the \PMlinkname{identity function}{IdentityMap}\, $f:t\mapsto{t}$.\, This yields from (1) the well-known ``memory formula''
                $$(x\!+\!y)(x\!-\!y) = x^2\!-\!y^2.$$

  \item $f''(t) = -k^2f(t)$\, with\, $k\neq 0$.\, According to the oddness one  obtains for the general solution the sine function\, $f:t\mapsto{C\sin{kt}}$.\, The special case\, $C = k = 1$\, means in (1) the \PMlinkescapetext{formula}
          $$\sin(x\!+\!y)\sin(x\!-\!y) = \sin^2x-\sin^2y,$$
which is easy to verify by using the \PMlinkname{addition and subtraction formulae}{AdditionFormula} of sine.

  \item $f''(t) = k^2f(t)$\, with\, $k\neq{0}$.\, According to the oddness we  obtain for the general solution the \PMlinkname{hyperbolic sine}{HyperbolicFunctions} function\, $f:t\mapsto{C\sinh{kt}}$.\, The special case\, $C = k = 1$\, gives from (1) the \PMlinkescapetext{formula}
          $$\sinh(x\!+\!y)\sinh(x\!-\!y) = \sinh^2x-\sinh^2y.$$

\end{enumerate}

The solution method of (1) is due to andik and perucho.
%%%%%
%%%%%
\end{document}
