\documentclass[12pt]{article}
\usepackage{pmmeta}
\pmcanonicalname{HartmanGrobmanTheorem}
\pmcreated{2013-03-22 13:18:37}
\pmmodified{2013-03-22 13:18:37}
\pmowner{jarino}{552}
\pmmodifier{jarino}{552}
\pmtitle{Hartman-Grobman theorem}
\pmrecord{4}{33816}
\pmprivacy{1}
\pmauthor{jarino}{552}
\pmtype{Theorem}
\pmcomment{trigger rebuild}
\pmclassification{msc}{34C99}

\endmetadata

% this is the default PlanetMath preamble.  as your knowledge
% of TeX increases, you will probably want to edit this, but
% it should be fine as is for beginners.

% almost certainly you want these
\usepackage{amssymb}
\usepackage{amsmath}
\usepackage{amsfonts}

% used for TeXing text within eps files
%\usepackage{psfrag}
% need this for including graphics (\includegraphics)
%\usepackage{graphicx}
% for neatly defining theorems and propositions
%\usepackage{amsthm}
% making logically defined graphics
%%%\usepackage{xypic}

% there are many more packages, add them here as you need them

% define commands here
\begin{document}
Consider the differential equation
\begin{equation}
x'=f(x)
\label{eq}
\end{equation}
where $f$ is a $C^1$ vector field. Assume that $x_0$ is a hyperbolic equilibrium of $f$. Denote $\Phi_t(x)$ the flow of (\ref{eq}) through $x$ at time $t$. Then there exists a homeomorphism $\varphi(x)=x+h(x)$ with $h$ bouded, such that
\[
\varphi\circ e^{tDf(x_0)}=\Phi_t\circ\varphi
\]
is a sufficiently small neighboorhood of $x_0$.

This fundamental theorem in the qualitative analysis of nonlinear differential equations states that, in a small neighborhood of $x_0$, the flow of the nonlinear equation (\ref{eq}) is qualitatively similar to that of the linear system $x'=Df(x_0)x$.
%%%%%
%%%%%
\end{document}
