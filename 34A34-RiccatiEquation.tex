\documentclass[12pt]{article}
\usepackage{pmmeta}
\pmcanonicalname{RiccatiEquation}
\pmcreated{2013-03-22 18:05:43}
\pmmodified{2013-03-22 18:05:43}
\pmowner{pahio}{2872}
\pmmodifier{pahio}{2872}
\pmtitle{Riccati equation}
\pmrecord{12}{40635}
\pmprivacy{1}
\pmauthor{pahio}{2872}
\pmtype{Result}
\pmcomment{trigger rebuild}
\pmclassification{msc}{34A34}
\pmclassification{msc}{34A05}
\pmsynonym{Riccati differential equation}{RiccatiEquation}
\pmrelated{BernoulliEquation}

\endmetadata

% this is the default PlanetMath preamble.  as your knowledge
% of TeX increases, you will probably want to edit this, but
% it should be fine as is for beginners.

% almost certainly you want these
\usepackage{amssymb}
\usepackage{amsmath}
\usepackage{amsfonts}

% used for TeXing text within eps files
%\usepackage{psfrag}
% need this for including graphics (\includegraphics)
%\usepackage{graphicx}
% for neatly defining theorems and propositions
 \usepackage{amsthm}
% making logically defined graphics
%%%\usepackage{xypic}

% there are many more packages, add them here as you need them

% define commands here

\theoremstyle{definition}
\newtheorem*{thmplain}{Theorem}

\begin{document}
The nonlinear differential equation
\begin{align}
\frac{dy}{dx} \;=\; f(x)+g(x)y+h(x)y^2
\end{align}
is called {\em Riccati equation}.\, If\, $h(x) \equiv 0$,\, it is a question of a linear differential equation; if\, $f(x) \equiv 0$,\, of a Bernoulli equation.\, There is no general method for integrating explicitely the equation (1), but 
via the substitution
$$y \;:=\; -\frac{w'(x)}{h(x)w(x)}$$
one can convert it to a \PMlinkescapetext{second order} homogeneous linear differential equation with non-constant coefficients.\\

If one can find a particular solution \,$y_0(x)$,\, then one can easily verify that the substitution
\begin{align}
y \;:=\; y_0(x)+\frac{1}{w(x)}
\end{align}
converts (1) to
\begin{align}
\frac{dw}{dx}+[g(x)\!+\!2h(x)y_0(x)]\,w+h(x) \;=\; 0,
\end{align}
which is a linear differential equation of first order with respect to the function \,$w =w(x)$.\\

\textbf{Example.}\, The Riccati equation
\begin{align}
\frac{dy}{x} \;=\; 3+3x^2y-xy^2
\end{align}
has the particular solution\, $y := 3x$.\, Solve the equation.

We substitute\, $y := 3x+\frac{1}{w(x)}$\, to (4), getting
$$\frac{dw}{dx}-3x^2w-x \;=\; 0.$$
For solving this \PMlinkname{first order equation}{LinearDifferentialEquationOfFirstOrder} we can put\, $w = uv$,\, $w' = uv'+u'v$,\, writing the equation as
\begin{align}
u\cdot(v'-3x^3v)+u'v \;:=\; x,
\end{align}
where we choose the value of the expression in parentheses equal to 0:
$$\frac{dv}{dx}-3x^2v \;=\; 0$$
After separation of variables and integrating, we obtain from here a solution\, $v = e^{x^3}$,\, which is set to the equation (5):
$$\frac{du}{dx}e^{x^3} \;=\; x$$  
Separating the variables yields
$$du \;=\; \frac{x}{e^{x^3}}\,dx$$
and integrating:
$$u \;=\; C+\int xe^{-x^3}\,dx.$$
Thus we have
$$w \;=\; w(x) \;=\; uv \;=\; e^{x^3}\left[C+\int xe^{-x^3}\,dx\right],$$
whence the general solution of the Riccati equation (4) is
$$\displaystyle y \;=\; 3x+\frac{e^{-x^3}}{C+\int xe^{-x^3}\,dx}.\\$$


It may be proved that if one knows three different solutions of Riccati equation (1), the each other solution may be expresses as a rational function of them.
 
 


%%%%%
%%%%%
\end{document}
