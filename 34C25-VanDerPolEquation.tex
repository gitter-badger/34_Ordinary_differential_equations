\documentclass[12pt]{article}
\usepackage{pmmeta}
\pmcanonicalname{VanDerPolEquation}
\pmcreated{2013-03-22 16:06:42}
\pmmodified{2013-03-22 16:06:42}
\pmowner{Daume}{40}
\pmmodifier{Daume}{40}
\pmtitle{van der Pol equation}
\pmrecord{13}{38177}
\pmprivacy{1}
\pmauthor{Daume}{40}
\pmtype{Definition}
\pmcomment{trigger rebuild}
\pmclassification{msc}{34C25}
\pmclassification{msc}{34C07}
\pmclassification{msc}{34-00}
\pmsynonym{van der Pol oscillator}{VanDerPolEquation}

\endmetadata

% this is the default PlanetMath preamble.  as your knowledge
% of TeX increases, you will probably want to edit this, but
% it should be fine as is for beginners.

% almost certainly you want these
\usepackage{amssymb}
\usepackage{amsmath}
\usepackage{amsfonts}
\usepackage{amsthm}

% used for TeXing text within eps files
%\usepackage{psfrag}
% need this for including graphics (\includegraphics)
\usepackage{graphicx}
%\usepackage{subfigure}
% making logically defined graphics
%%%\usepackage{xypic} 

% there are many more packages, add them here as you need them

% define commands here

% The below lines should work as the command
% \renewcommand{\bibname}{References}
% without creating havoc when rendering an entry in
% the page-image mode.
\makeatletter
\@ifundefined{bibname}{}{\renewcommand{\bibname}{References}}
\makeatother

\newtheorem{thm}{Theorem}
\newtheorem{defn}{Definition}
\newtheorem{prop}{Proposition}
\newtheorem{lemma}{Lemma}
\newtheorem{cor}{Corollary}
\begin{document}
\PMlinkescapeword{properties}
\PMlinkescapeword{circuit}
\PMlinkescapeword{second order}
\PMlinkescapeword{order}
\PMlinkescapeword{planar}
\PMlinkescapeword{term}
\PMlinkescapeword{model}
\PMlinkescapeword{representation}
\PMlinkescapeword{focus}

In 1920 the Dutch physicist Balthasar van der Pol studied a differential equation that describes the circuit of a vacuum tube. 
It has been used
to model other phenomenon such as the human heartbeat by
Johannes van der Mark\cite{C}.\\

The \emph{van der Pol equation} 
equation is a case of a Lienard system and is expressed as
a second order ordinary differential equation
$$\frac{d^2x}{dt^2}-\mu(1-x^2)\frac{dx}{dt}+x=0$$
or a first order planar ordinary differential equation
\begin{eqnarray*}
\dot{x} & = & y + \mu(x-x^3)\\
\dot{y} & = & -x
\end{eqnarray*}
where $\mu$ is a real parameter.  
The parameter $\mu$ is usually considered to be positive since
the the term $-\mu(1-x^2)$ adds to the model a nonlinear damping. \cite{C}

\textbf{Properties:}
\begin{itemize}
\item If $\mu=0$ then the origin is a center. In fact, if $\mu=0$ then
$$\frac{d^2x}{dt^2}+x=0$$ 
and if we suppose that the initial condition are
$(x_0,\dot{x}_0)$ then the solution to the system is 
$$x(t)=x_0\cos t + \dot{x}_0\sin t.$$ All
solutions except the origin are periodic and circles.  See phase portrait below. 
\item If $\mu>0$ the system has a unique limit cycle, and the limit
cycle is attractive.  This follows directly from Lienard's theorem. \cite{P}
\item The system is sometimes given under the form
\begin{eqnarray*}
\dot{X} &=& -Y\\
\dot{Y} &=& X + \mu(1-X^2)Y
\end{eqnarray*}
which equivalent to the previous planar system 
under the change of coordinate $(X, Y)= (\sqrt{3}x,-\sqrt{3}(y+\mu(x-x^3)))$.\cite{C}
\end{itemize}

\textbf{Example:}\\
The geometric representation of the phase portrait is done 
by taking initial condition from
an equally spaced grid and calculating the solution for positive and
negative time.  

For the parameter $\mu=1$, the system has 
an attractive limit cycle and the origin is a repulsive focus.
\begin{center}
\includegraphics[scale=0.5]{vanderpol_mu1.eps}\\
\small Phase portrait when $\mu=1$.
\end{center}
When the parameter $\mu=0$ the origin is a center.
\begin{center}
\includegraphics[scale=0.5]{vanderpol_mu0.eps}\\
\small Phase portrait when $\mu=0$.
\end{center}
For the parameter
$\mu=-1$, the system has a repulsive limit cycle and the origin is
an attractive focus.
\begin{center}
\includegraphics[scale=0.5]{vanderpol_mum1.eps}\\
\small Phase portrait when $\mu=-1$.
\end{center}

\begin{thebibliography}{1}
\bibitem[C]{C}
{\scshape Chicone, Carmen},
\emph{Ordinary Differential Equations with Applications},
Springer, New York, 1999.

\bibitem[P]{P}
{\scshape Perko, Lawrence},
\emph{Differential Equations and Dynamical Systems},
Springer, New York, 2001.
\end{thebibliography}
%%%%%
%%%%%
\end{document}
