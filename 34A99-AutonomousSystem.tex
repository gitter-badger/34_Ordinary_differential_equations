\documentclass[12pt]{article}
\usepackage{pmmeta}
\pmcanonicalname{AutonomousSystem}
\pmcreated{2013-03-22 13:37:26}
\pmmodified{2013-03-22 13:37:26}
\pmowner{Daume}{40}
\pmmodifier{Daume}{40}
\pmtitle{autonomous system}
\pmrecord{6}{34260}
\pmprivacy{1}
\pmauthor{Daume}{40}
\pmtype{Definition}
\pmcomment{trigger rebuild}
\pmclassification{msc}{34A99}
\pmsynonym{autonomous}{AutonomousSystem}
\pmsynonym{autonomous equation}{AutonomousSystem}
\pmsynonym{nonautonomous}{AutonomousSystem}
\pmsynonym{nonautonomous equation}{AutonomousSystem}
\pmrelated{TimeInvariant}
\pmrelated{SystemDefinitions}
\pmdefines{nonautonomous system}

\endmetadata

% this is the default PlanetMath preamble.  as your knowledge
% of TeX increases, you will probably want to edit this, but
% it should be fine as is for beginners.

% almost certainly you want these
\usepackage{amssymb}
\usepackage{amsmath}
\usepackage{amsfonts}

% used for TeXing text within eps files
%\usepackage{psfrag}
% need this for including graphics (\includegraphics)
%\usepackage{graphicx}
% for neatly defining theorems and propositions
%\usepackage{amsthm}
% making logically defined graphics
%%%\usepackage{xypic} 

% there are many more packages, add them here as you need them

% define commands here
\begin{document}
A system of ordinary differential equation is \emph{autonomous} when it does not depend on time \textit{(does not depend on the independent variable)} i.e. $\dot{x}=f(x)$.  In contrast \emph{nonautonomous} is when the system of ordinary differential equation does depend on time \textit{(does depend on the independent variable)} i.e. $\dot{x}=f(x,t)$.\\

It can be noted that every nonautonomous system can be converted to an autonomous system by adding a dimension. i.e. If $\dot{\textbf{x}}=\textbf{f}(\textbf{x},t)$ $\textbf{x} \in \mathbb{R}^n$ then it can be written as an autonomous system with $\textbf{x} \in \mathbb{R}^{n+1}$ and by doing a substitution with $x_{n+1} = t$ and $\dot{x}_{n+1}=1$.
%%%%%
%%%%%
\end{document}
