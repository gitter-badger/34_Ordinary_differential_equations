\documentclass[12pt]{article}
\usepackage{pmmeta}
\pmcanonicalname{DerivativesOfSolutionOfFirstOrderODE}
\pmcreated{2013-03-22 18:59:14}
\pmmodified{2013-03-22 18:59:14}
\pmowner{pahio}{2872}
\pmmodifier{pahio}{2872}
\pmtitle{derivatives of solution of first order ODE}
\pmrecord{10}{41853}
\pmprivacy{1}
\pmauthor{pahio}{2872}
\pmtype{Theorem}
\pmcomment{trigger rebuild}
\pmclassification{msc}{34A12}
\pmclassification{msc}{34-00}
\pmrelated{SolutionsOfOrdinaryDifferentialEquation}
\pmrelated{InflexionPoint}

% this is the default PlanetMath preamble.  as your knowledge
% of TeX increases, you will probably want to edit this, but
% it should be fine as is for beginners.

% almost certainly you want these
\usepackage{amssymb}
\usepackage{amsmath}
\usepackage{amsfonts}

% used for TeXing text within eps files
%\usepackage{psfrag}
% need this for including graphics (\includegraphics)
%\usepackage{graphicx}
% for neatly defining theorems and propositions
 \usepackage{amsthm}
% making logically defined graphics
%%%\usepackage{xypic}

% there are many more packages, add them here as you need them

% define commands here

\theoremstyle{definition}
\newtheorem*{thmplain}{Theorem}

\begin{document}
\PMlinkescapeword{order}

Suppose that $f$ is a continuously differentiable function defined on an open subset $E$ of $\mathbb{R}^2$, i.e. it has on $E$ the continuous partial derivatives \,$f_x'(x,\,y)$\, and\, $f_y'(x,\,y)$.

If $y(x)$ is a solution of the \PMlinkescapetext{first order} ordinary differential equation
\begin{align}
\frac{dy}{dx} \;=\; f(x,\,y),
\end{align}
then we have
\begin{align}
y'(x) \;=\; f(x,\,y(x)), 
\end{align}
\begin{align}
y''(x) \;=\; f_x'(x,\,y(x))+f_y'(x,\,y(x))\,y'(x)
\end{align}
(see the \PMlinkid{general chain rule}{2798}).\, Thus there exists on $E$ the second derivative $y''(x)$ which is also continuous.\, More generally, we can infer the

\textbf{Theorem.}\, If\, $f(x,\,y)$\, has in $E$ the continuous partial derivatives up to the order $n$, then any solution $y(x)$ of the differential equation (1) has on $E$ the continuous derivatives $y^{(i)}(x)$ up to the \PMlinkname{order}{OrderOfDerivative} $n\!+\!1$.\\

\textbf{Note 1.}\, The derivatives $y^{(i)}(x)$ are got from the equation (1) via succesive differentiations.\, Two first ones are (2) and (3), and the next two ones, with a simpler notation:
$$y''' \;=\; f_{xx}''+2f_{xy}''y'+f_{yy}''y'^2+f_y'y'',$$
$$y^{(4)} \;=\; f_{xxx}'''+3f_{xxy}'''y'+3f_{xyy}'''y'^2+f_{yyy}'''y'^3+3f_{xy}''y''+3f_{yy}''y'y''+f_y'y'''$$

\textbf{Note 2.}\, It follows from (3) that the curve
\begin{align}
f_x'(x,\,y)+f_y'(x,\,y)f(x,\,y) \;=\; 0
\end{align}
is the locus of the inflexion points of the integral curves of (1), or more exactly, the locus of the points where the integral curves have with their tangents a \PMlinkname{contact of order}{OrderOfContact} more than one.\, The curve (4) is also the locus of the points of tangency of the integral curves and their isoclines.

\begin{thebibliography}{9}
\bibitem{3L}{\sc E. Lindel\"of:} {\em Differentiali- ja integralilasku III 1}.\, Mercatorin Kirjapaino Osakeyhti\"o, Helsinki (1935).
\end{thebibliography}




%%%%%
%%%%%
\end{document}
