\documentclass[12pt]{article}
\usepackage{pmmeta}
\pmcanonicalname{MethodsOfSolvingODEs}
\pmcreated{2013-03-22 17:52:02}
\pmmodified{2013-03-22 17:52:02}
\pmowner{invisiblerhino}{19637}
\pmmodifier{invisiblerhino}{19637}
\pmtitle{methods of solving ODEs}
\pmrecord{9}{40346}
\pmprivacy{1}
\pmauthor{invisiblerhino}{19637}
\pmtype{Topic}
\pmcomment{trigger rebuild}
\pmclassification{msc}{34-00}

\endmetadata

% this is the default PlanetMath preamble.  as your knowledge
% of TeX increases, you will probably want to edit this, but
% it should be fine as is for beginners.

% almost certainly you want these
\usepackage{amssymb}
\usepackage{amsmath}
\usepackage{amsfonts}

% used for TeXing text within eps files
%\usepackage{psfrag}
% need this for including graphics (\includegraphics)
%\usepackage{graphicx}
% for neatly defining theorems and propositions
%\usepackage{amsthm}
% making logically defined graphics
%%%\usepackage{xypic}

% there are many more packages, add them here as you need them

% define commands here

\begin{document}
There are many different methods of solving ordinary differential equations. These are tabulated here and divided into sections for linear and non-linear ODEs. These sections are further subdivided into exact, numerical and quantitative methods.
\section{Linear ODEs}
\subsection{Exact methods}
\begin{itemize}
\item General solution of linear differential equation
\item Linear differential equation of first order
\item Method of separation of variables
\item Method of integrating factors
\item Method of undetermined coefficients
\item Frobenius' Method
\item Method of Green's functions
\end{itemize}
\subsection{Numerical methods}
\begin{itemize}
\item Runge-Kutta method
\item Heun method
\item Rayleigh-Ritz method
\end{itemize}
\subsection{Quantitative methods}
\section{Non-linear ODEs}
%%%%%
%%%%%
\end{document}
