\documentclass[12pt]{article}
\usepackage{pmmeta}
\pmcanonicalname{SecondOrderOrdinaryDifferentialEquation}
\pmcreated{2013-03-22 18:35:39}
\pmmodified{2013-03-22 18:35:39}
\pmowner{pahio}{2872}
\pmmodifier{pahio}{2872}
\pmtitle{second order ordinary differential equation}
\pmrecord{5}{41322}
\pmprivacy{1}
\pmauthor{pahio}{2872}
\pmtype{Topic}
\pmcomment{trigger rebuild}
\pmclassification{msc}{34A05}
\pmdefines{normal system}
\pmdefines{normal system of second order}

% this is the default PlanetMath preamble.  as your knowledge
% of TeX increases, you will probably want to edit this, but
% it should be fine as is for beginners.

% almost certainly you want these
\usepackage{amssymb}
\usepackage{amsmath}
\usepackage{amsfonts}

% used for TeXing text within eps files
%\usepackage{psfrag}
% need this for including graphics (\includegraphics)
%\usepackage{graphicx}
% for neatly defining theorems and propositions
 \usepackage{amsthm}
% making logically defined graphics
%%%\usepackage{xypic}

% there are many more packages, add them here as you need them

% define commands here

\theoremstyle{definition}
\newtheorem*{thmplain}{Theorem}

\begin{document}
A second order ordinary differential equation\, $F(x,\,y,\,\frac{dy}{dx},\,\frac{d^2y}{dx^2}) = 0$\, can often be written in the form
\begin{align}
\frac{d^2y}{dx^2} \;=\; f\left(x,\,y,\,\frac{dy}{dx}\right).
\end{align}
By its general solution one means a function \; $x \mapsto y = y(x)$\; which is at least on an interval twice differentiable and satisfies
\[y''(x) \;\equiv\; f(x,\,y(x),\,y'(x)).\]
By setting\, $\frac{dy}{dx} := z$,\, one has\, $\frac{d^2y}{dx^2} = \frac{dz}{dx}$,\, and the equation (1) reads\, 
$\frac{dz}{dx} = f(x,\,y,\,z)$.\, It is easy to see that solving (1) is \PMlinkname{equivalent}{Equivalent3} with solving the system of simultaneous \PMlinkname{first order}{ODE} differential equations
\begin{align}
\begin{cases}
\frac{dy}{dx} = z,\\
\frac{dz}{dx} = f(x,\,y,\,z),
\end{cases}
\end{align}
the so-called {\em normal system} of (1).\\

The system (2) is a special case of the general {\em normal system of second order}, which has the form
\begin{align}
\begin{cases}
\frac{dy}{dx} = \varphi(x,\,y,\,z),\\
\frac{dz}{dx} = \psi(x,\,y,\,z),
\end{cases}
\end{align}
where $y$ and $z$ are unknown functions of the variable $x$.\, The existence theorem of the solution
\begin{align}
\begin{cases}
y = y(x),\\
z = z(x)
\end{cases}
\end{align}
is as follows; cf. the \PMlinkname{Picard--Lindel\"of theorem}{PicardsTheorem2}.

\textbf{Theorem.}\, If the functions $\varphi$ and $\psi$ are continuous and have continuous partial derivatives with respect to $y$ and $z$ in a neighbourhood of a point \,$(x_0,\,y_0,\,z_0)$,\, then the normal system (3) has one and (as long as $|x\!-\!x_0|$ does not exceed a certain \PMlinkescapetext{bound}) only one solution (4)
which satisfies the initial conditions \, $y(x_0) = y_0,\;\, z(x_0) = z_0$.\; The functions (4) are continuously differentiable in a neighbourhood of $x_0$.

\begin{thebibliography}{9}
\bibitem{3L}{\sc E. Lindel\"of:} {\em Differentiali- ja integralilasku III 1}.\, Mercatorin Kirjapaino Osakeyhti\"o, Helsinki (1935).
\end{thebibliography}


%%%%%
%%%%%
\end{document}
