\documentclass[12pt]{article}
\usepackage{pmmeta}
\pmcanonicalname{ClairautsEquation}
\pmcreated{2015-02-04 11:20:02}
\pmmodified{2015-02-04 11:20:02}
\pmowner{pahio}{2872}
\pmmodifier{pahio}{2872}
\pmtitle{Clairaut's equation}
\pmrecord{21}{37024}
\pmprivacy{1}
\pmauthor{pahio}{2872}
\pmtype{Derivation}
\pmcomment{trigger rebuild}
\pmclassification{msc}{34C05}
\pmsynonym{Clairaut differential equation}{ClairautsEquation}
\pmrelated{DAlembertsEquation}
\pmrelated{FamousCurvesInThePlane}
\pmrelated{IndexOfDifferentialEquations}
\pmrelated{PerimeterOfAstroid}
\pmrelated{SingularSolution}
\pmrelated{DerivativeAsParameterForSolvingDifferentialEquations}
\pmdefines{astroid}

\endmetadata

% this is the default PlanetMath preamble.  as your knowledge
% of TeX increases, you will probably want to edit this, but
% it should be fine as is for beginners.

% almost certainly you want these
\usepackage{amssymb}
\usepackage{amsmath}
\usepackage{amsfonts}

% used for TeXing text within eps files
%\usepackage{psfrag}
% need this for including graphics (\includegraphics)
\usepackage{graphicx}
% for neatly defining theorems and propositions
 \usepackage{amsthm}
% making logically defined graphics
%%%\usepackage{xypic}

% there are many more packages, add them here as you need them

% define commands here

\theoremstyle{definition}
\newtheorem*{thmplain}{Theorem}

\begin{document}
The ordinary differential equation
\begin{align}
y \;=\; x\frac{dy}{dx}+\psi\left(\frac{dy}{dx}\right),
\end{align}
where $\psi$ is a given differentiable real function, is called {\em Clairaut's equation}.

For solving the equation we use an auxiliary variable\, $p =: \frac{dy}{dx}$\, and write (1) as
$$y \;=\; px+\psi(p).$$
Differentiating this equation gives
$$p  \;=\; x\frac{dp}{dx}+p+\psi'(p)\frac{dp}{dx},$$
or
$$[x+\psi'(p)]\frac{dp}{dx} \;=\; 0.$$
The zero rule of product now yields the alternatives
\begin{align}
      \frac{dp}{dx} \;=\; 0
\end{align}
and
\begin{align}
        x+\psi'(p) \;=\; 0.
\end{align}
Integrating (2) we get\, $p = C$ (\PMlinkescapetext{constant}), and substituting this in (1) gives the general solution
\begin{align}
      y \;=\; Cx+\psi(C)
\end{align}
which presents a family of straight lines.\\

If (3) allows to solve $p$ in \PMlinkescapetext{terms} of $x$,\, $p = p(x)$,\, we can write (1) as
\begin{align}
      y \;=\; xp(x)+\psi(p(x)),
\end{align}
which is easy to see satisfying (1).\, The solution (5) may not be gotten from (4) using any value of $C$.\, It is a singular solution which may be obtained by eliminating the parameter $p$ from the equations
   $$y \;=\; px+\psi(p), \qquad x+\psi'(p) \;=\; 0.$$
Thus the singular solution presents the envelope of the family (4).\\

\textbf{Example.}\, The Clairaut's equation
  $$y \;=\; x\frac{dy}{dx}+\frac{a\frac{dy}{dx}}{\sqrt{1+(\frac{dy}{dx})^2}}$$
has the general solution
     $$y \;=\; Cx+\frac{Ca}{\sqrt{1\!+\!C^2}}$$
and the singular solution
$$\begin{cases}
                x \;=\; -\frac{a}{(1\!+\!p^2)^{3/2}},\\
                y \;=\; -\frac{ap^3}{(1+p^2)^{3/2}}\\
\end{cases}$$
in a parametric form.\, Eliminating the parametre $p$ yields the form
   $$\sqrt[3]{x^2}+\sqrt[3]{y^2} \;=\; \sqrt[3]{a^2},$$
which can be recognized to be the equation of an {\em astroid}.\, The envelope (see ``\PMlinkname{determining envelope}{DeterminingEnvelope}'') of the lines is only the left half of this curve ($x \leqq 0$).\, The usual parametric \PMlinkescapetext{presentation} of the astroid is\, $x = a\cos^3\varphi$,\, $y = a\sin^3\varphi$\, 
($0 \leqq \varphi < 2\pi$).

\begin{center}
\includegraphics{clairaut2}
\end{center}

\begin{thebibliography}{9}
\bibitem{NP}{\sc N. Piskunov:} {\em Diferentsiaal- ja integraalarvutus k\~{o}rgematele tehnilistele \~{o}ppeasutustele}.\, -- Kirjastus Valgus, Tallinn  (1966).
\end{thebibliography}

%%%%%
%%%%%
\end{document}
