\documentclass[12pt]{article}
\usepackage{pmmeta}
\pmcanonicalname{HomogeneousLinearDifferentialEquation}
\pmcreated{2014-02-27 10:07:04}
\pmmodified{2014-02-27 10:07:04}
\pmowner{pahio}{2872}
\pmmodifier{pahio}{2872}
\pmtitle{homogeneous linear differential equation}
\pmrecord{3}{88039}
\pmprivacy{1}
\pmauthor{pahio}{2872}
\pmtype{Definition}
\pmclassification{msc}{34A05}

\endmetadata

% this is the default PlanetMath preamble.  as your knowledge
% of TeX increases, you will probably want to edit this, but
% it should be fine as is for beginners.

% almost certainly you want these
\usepackage{amssymb}
\usepackage{amsmath}
\usepackage{amsfonts}

% need this for including graphics (\includegraphics)
\usepackage{graphicx}
% for neatly defining theorems and propositions
\usepackage{amsthm}

% making logically defined graphics
%\usepackage{xypic}
% used for TeXing text within eps files
%\usepackage{psfrag}

% there are many more packages, add them here as you need them

% define commands here

\begin{document}
The linear differential equation
\begin{align}
a_n(x)y^{(n)}+a_{n-1}(x)y^{(n-1)}+\ldots+a_1(x)y'+a_0(x)y \;=\; b(x)
\end{align}
is called 
\PMlinkname{{\it homogeneous}}{HomogeneousLinearDifferentialEquation} 
iff\, $b(x) \equiv 0$.\, If\, $b(x) \not\equiv 0$, 
the equation (1) is {\it inhomogeneous}.\\
If (1) is \PMlinkname{homogeneous}{HomogeneousLinearDifferentialEquation}, 
then the sum of any solutions is a solution and any solution 
multiplied by a constant is a solution.\\

The special case 
$$c_nx^ny^{(n)}+c_{n-1}x^{n-1}y^{(n-1)}+\ldots+c_1xy'+c_0y \;=\; 0$$
of (1), where the $c_i$'s are constants, can via the 
substitution\, $x = e^t$\, be transformed into a homogeneous 
linear differential equation of the same order but with 
constant coefficients.\\

\end{document}
