\documentclass[12pt]{article}
\usepackage{pmmeta}
\pmcanonicalname{VariationOfParameters}
\pmcreated{2013-03-22 12:39:16}
\pmmodified{2013-03-22 12:39:16}
\pmowner{rspuzio}{6075}
\pmmodifier{rspuzio}{6075}
\pmtitle{variation of parameters}
\pmrecord{8}{32921}
\pmprivacy{1}
\pmauthor{rspuzio}{6075}
\pmtype{Theorem}
\pmcomment{trigger rebuild}
\pmclassification{msc}{34A30}
\pmclassification{msc}{34A05}
\pmsynonym{variation of constants}{VariationOfParameters}

% this is the default PlanetMath preamble.  as your knowledge
% of TeX increases, you will probably want to edit this, but
% it should be fine as is for beginners.

% almost certainly you want these
\usepackage{amssymb}
\usepackage{amsmath}
\usepackage{amsfonts}

% used for TeXing text within eps files
%\usepackage{psfrag}
% need this for including graphics (\includegraphics)
%\usepackage{graphicx}
% for neatly defining theorems and propositions
%\usepackage{amsthm}
% making logically defined graphics
%%%\usepackage{xypic}

% there are many more packages, add them here as you need them

% define commands here
\begin{document}
The method of \emph{variation of parameters} is a way of finding a particular
solution to a nonhomogeneous linear differential equation.

Suppose that we have an $n$th order linear differential operator
\begin{equation*}
 L[y] := y^{(n)} + p_1(t) y^{(n-1)} + \cdots + p_n(t) y ,
\end{equation*}
and a corresponding nonhomogeneous differential equation
\begin{equation}
 \label{nonhom-diffeq}
 L[y] = g(t).
\end{equation}
Suppose that we know a fundamental set of solutions
$y_1,y_2,\ldots,y_n$ of the corresponding homogeneous differential equation
$L[y_c]=0$. The general solution of the homogeneous equation is
\begin{equation*}
 y_c(t) = c_1 y_1(t) + c_2 y_2(t) + \cdots + c_n y_n(t),
\end{equation*}
where $c_1,c_2,\ldots,c_n$ are constants.
The general solution to the nonhomogeneous equation $L[y]=g(t)$ is then
\begin{equation*}
 y(t) = y_c(t) + Y(t),
\end{equation*}
where $Y(t)$ is a particular solution which satisfies $L[Y]=g(t)$, and the
constants $c_1,c_2,\ldots,c_n$ are chosen to satisfy the appropriate
boundary conditions or initial conditions.

The key step in using variation of parameters is to suppose that the
particular solution is given by
\begin{equation}
 \label{part-sol-form}
 Y(t) = u_1(t) y_1(t) + u_2(t) y_2(t) + \cdots + u_n(t) y_n(t),
\end{equation}
where $u_1(t),u_2(t),\ldots,u_n(t)$ are as yet to be determined functions
(hence the name \emph{variation of parameters}). To find
these $n$ functions we need a set of $n$ independent equations.
One obvious condition is that the proposed ansatz satisfies Eq.
\eqref{nonhom-diffeq}. Many possible additional conditions are possible,
we choose the ones that make further calculations easier. Consider the
following set of $n-1$ conditions
\begin{eqnarray*}
 y_1 u_1' + y_2 u_2' + \cdots + y_n u_n' &=& 0 \\
 y_1' u_1' + y_2' u_2' + \cdots + y_n' u_n' &=& 0 \\
  &\vdots \\
 y_1^{(n-2)} u_1' + y_2^{(n-2)} u_2' + \cdots + y_n^{(n-2)} u_n' &=& 0.
\end{eqnarray*}
Now, substituting Eq.~\eqref{part-sol-form} into $L[Y]=g(t)$ and using the
above conditions, we can get another equation
\begin{equation*}
 y_1^{(n-1)} u_1' + y_2^{(n-1)} u_2' + \cdots + y_n^{(n-1)} u_n' = g .
\end{equation*}

So we have a system of $n$ equations for $u_1',u_2',\ldots,u_n'$ which
we can solve using Cramer's rule:
\begin{equation*}
 u_m'(t) = \frac{g(t) W_m(t)}{W(t)}, \quad m=1,2,\ldots,n .
\end{equation*}
Such a solution always exists since the Wronskian $W=W(y_1,y_2,\ldots,y_n)$
of the system is nowhere zero, because the $y_1,y_2,\ldots,y_n$ form a
fundamental set of solutions. Lastly the term $W_m$ is the Wronskian
determinant with the $m$th column replaced by the column
$(0,0,\ldots,0,1)$.

Finally the particular solution can be written explicitly as
\begin{equation*}
 Y(t) = \sum_{m=1}^n y_m(t) \int \frac{g(t) W_m(t)}{W(t)} dt .
\end{equation*}

\begin{thebibliography}{1}

\bibitem{Boyce}
 W.~E.~Boyce and R.~C.~DiPrima.
 \textit{Elementary Differential Equations and Boundary Value Problems}
 John Wiley \& Sons, 6th edition, 1997.

\end{thebibliography}
%%%%%
%%%%%
\end{document}
