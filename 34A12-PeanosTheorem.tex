\documentclass[12pt]{article}
\usepackage{pmmeta}
\pmcanonicalname{PeanosTheorem}
\pmcreated{2013-03-22 15:20:21}
\pmmodified{2013-03-22 15:20:21}
\pmowner{Daume}{40}
\pmmodifier{Daume}{40}
\pmtitle{Peano's theorem}
\pmrecord{6}{37157}
\pmprivacy{1}
\pmauthor{Daume}{40}
\pmtype{Theorem}
\pmcomment{trigger rebuild}
\pmclassification{msc}{34A12}

\endmetadata

% this is the default PlanetMath preamble.  as your knowledge
% of TeX increases, you will probably want to edit this, but
% it should be fine as is for beginners.

% almost certainly you want these
\usepackage{amssymb}
\usepackage{amsmath}
\usepackage{amsfonts}
\usepackage{amsthm}

% used for TeXing text within eps files
%\usepackage{psfrag}
% need this for including graphics (\includegraphics)
%\usepackage{graphicx}
% making logically defined graphics
%%%\usepackage{xypic} 

% there are many more packages, add them here as you need them

% define commands here

% The below lines should work as the command
% \renewcommand{\bibname}{References}
% without creating havoc when rendering an entry in
% the page-image mode.
\makeatletter
\@ifundefined{bibname}{}{\renewcommand{\bibname}{References}}
\makeatother

\newtheorem{thm}{Theorem}
\newtheorem{defn}{Definition}
\newtheorem{prop}{Proposition}
\newtheorem{lemma}{Lemma}
\newtheorem{cor}{Corollary}
\begin{document}
\PMlinkescapeword{planar}
\PMlinkescapeword{domain}
If
\begin{eqnarray}\label{eq}
\frac{dy}{dx}&=&f(x,y)
\end{eqnarray}
is an ordinary differential equation where $f(x,y)$ is continuous 
on a planar domain $E$, then (\ref{eq}) has at least one integral 
curve for each $(x_0,y_0)$ of $E$.\cite{KF,T}

\begin{thebibliography}{1}
\bibitem[KF]{KF} 
{\scshape Kolmogorov, A.N. \& Fomin, S.V.},
\emph{Introductory Real Analysis, Translated \& Edited by Richard A. Silverman}, Dover Publications, Inc. New York, 1970.
\bibitem[T]{T}
{\scshape Teschl, Gerald},
\emph{\PMlinkescapetext{Ordinary Differential Equations and Dynamical System}}
\PMlinkexternal{http://www.mat.univie.ac.at/~gerald/ftp/book-ode/index.html}{http://www.mat.univie.ac.at/~gerald/ftp/book-ode/index.html}, 2004.
\end{thebibliography}
%%%%%
%%%%%
\end{document}
