\documentclass[12pt]{article}
\usepackage{pmmeta}
\pmcanonicalname{GrammianDeterminant}
\pmcreated{2013-03-22 17:37:33}
\pmmodified{2013-03-22 17:37:33}
\pmowner{slider142}{78}
\pmmodifier{slider142}{78}
\pmtitle{Grammian determinant}
\pmrecord{6}{40046}
\pmprivacy{1}
\pmauthor{slider142}{78}
\pmtype{Definition}
\pmcomment{trigger rebuild}
\pmclassification{msc}{34A12}
\pmrelated{WronskianDeterminant}
\pmrelated{GramDeterminant}

% this is the default PlanetMath preamble.  as your knowledge
% of TeX increases, you will probably want to edit this, but
% it should be fine as is for beginners.

% almost certainly you want these
\usepackage{amssymb}
\usepackage{amsmath}
\usepackage{amsfonts}

% used for TeXing text within eps files
%\usepackage{psfrag}
% need this for including graphics (\includegraphics)
%\usepackage{graphicx}
% for neatly defining theorems and propositions
%\usepackage{amsthm}
% making logically defined graphics
%%%\usepackage{xypic} 

% there are many more packages, add them here as you need them

% define commands here

\begin{document}
The Grammian determinant provides a necessary and sufficient method of determining whether a set of continuous functions ${f_1, f_2, \dotsc, f_n}$ is linearly independent on an interval $I = [a, b]$ with respect to the inner product 
$$\langle f_i | f_j\rangle = \int_I f_if_j$$ 
It is defined as:
\[
G(f_1, f_2, \dotsc, f_n) = \left\lvert\begin{array}{@{}cccc@{}}
\int_I (f_1)^2 & \int_I f_1f_2 & \cdots & \int_I f_1f_n\\
\int_I f_2f_1 & \int_I (f_2)^2 & \cdots & \int_I f_2f_n\\
\vdots & \vdots & \ddots & \vdots\\
\int_I f_nf_1 & \int_I f_nf_2 & \cdots & \int_I (f_n)^2\\
\end{array}\right\rvert
\]
If the functions are continuous on $I$, then $G = 0$ if and only if the set of functions is linearly dependent. Note that the Grammian determinant is a special case of the more general Gram determinant.
%%%%%
%%%%%
\end{document}
