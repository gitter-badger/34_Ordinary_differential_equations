\documentclass[12pt]{article}
\usepackage{pmmeta}
\pmcanonicalname{LienardSystem}
\pmcreated{2013-03-22 15:20:45}
\pmmodified{2013-03-22 15:20:45}
\pmowner{Daume}{40}
\pmmodifier{Daume}{40}
\pmtitle{Lienard system}
\pmrecord{6}{37166}
\pmprivacy{1}
\pmauthor{Daume}{40}
\pmtype{Definition}
\pmcomment{trigger rebuild}
\pmclassification{msc}{34-00}
\pmdefines{Lienard equation}

\endmetadata

% this is the default PlanetMath preamble.  as your knowledge
% of TeX increases, you will probably want to edit this, but
% it should be fine as is for beginners.

% almost certainly you want these
\usepackage{amssymb}
\usepackage{amsmath}
\usepackage{amsfonts}
\usepackage{amsthm}

% used for TeXing text within eps files
%\usepackage{psfrag}
% need this for including graphics (\includegraphics)
%\usepackage{graphicx}
% making logically defined graphics
%%%\usepackage{xypic} 

% there are many more packages, add them here as you need them

% define commands here

% The below lines should work as the command
% \renewcommand{\bibname}{References}
% without creating havoc when rendering an entry in
% the page-image mode.
\makeatletter
\@ifundefined{bibname}{}{\renewcommand{\bibname}{References}}
\makeatother

\newtheorem{thm}{Theorem}
\newtheorem{defn}{Definition}
\newtheorem{prop}{Proposition}
\newtheorem{lemma}{Lemma}
\newtheorem{cor}{Corollary}
\begin{document}
\PMlinkescapeword{equivalent}

A \emph{Lienard system} is a planar ordinary differential equation
\begin{eqnarray*}
\dot{x} & = & y -f(x)\\
\dot{y} & = & -g(x)
\end{eqnarray*}
with conditions on the smoothness of $f$ and $g$.  It is equivalent 
to the following second order ordinary differential equation
$$\ddot{x}+f'(x)\dot{x}+g(x)=0.$$
\textbf{Example:}
\begin{itemize}
\item van der Pol equation
\end{itemize}

\begin{thebibliography}{1}
\bibitem[P]{P}
{\scshape Perko, Lawrence},
\emph{Differential Equations and Dynamical Systems},
Springer, New York, 2001.
\end{thebibliography}
%%%%%
%%%%%
\end{document}
