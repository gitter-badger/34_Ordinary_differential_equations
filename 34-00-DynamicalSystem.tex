\documentclass[12pt]{article}
\usepackage{pmmeta}
\pmcanonicalname{DynamicalSystem}
\pmcreated{2013-03-22 14:06:25}
\pmmodified{2013-03-22 14:06:25}
\pmowner{PrimeFan}{13766}
\pmmodifier{PrimeFan}{13766}
\pmtitle{dynamical system}
\pmrecord{14}{35508}
\pmprivacy{1}
\pmauthor{PrimeFan}{13766}
\pmtype{Definition}
\pmcomment{trigger rebuild}
\pmclassification{msc}{34-00}
\pmclassification{msc}{37-00}
\pmsynonym{supercategorical dynamics}{DynamicalSystem}
%\pmkeywords{dynamical systems}
\pmrelated{SystemDefinitions}
\pmrelated{GroupoidCDynamicalSystem}
\pmrelated{CategoricalDynamics}
\pmrelated{Bifurcation}
\pmrelated{ChaoticDynamicalSystem}
\pmrelated{IndexOfCategories}
\pmdefines{planar dynamical system}

% this is the default PlanetMath preamble.  as your knowledge
% of TeX increases, you will probably want to edit this, but
% it should be fine as is for beginners.

% almost certainly you want these
\usepackage{amssymb}
\usepackage{amsmath}
\usepackage{amsfonts}

% used for TeXing text within eps files
%\usepackage{psfrag}
% need this for including graphics (\includegraphics)
%\usepackage{graphicx}
% for neatly defining theorems and propositions
%\usepackage{amsthm}
% making logically defined graphics
%%%\usepackage{xypic} 

% there are many more packages, add them here as you need them

% define commands here
\begin{document}
A \emph{dynamical system} on $X$ where $X$ is an open subset of $\mathbb{R}^n$ is a differentiable map
$$\phi: \mathbb{R}\times X \to X$$
where
$$\phi (t,\mathbf{x}) = \phi_t (\mathbf{x})$$
satisfies
\begin{itemize}
\item[i] $\phi_0(\mathbf{x}) = \mathbf{x}$ for all $\mathbf{x}\in X$ \textit{(the identity function)}
\item[ii] $\phi_t \circ \phi_s (\mathbf{x}) = \phi_{t+s}(\mathbf{x})$ for all $s,t \in \mathbb{R}$  \textit{(composition)}
\end{itemize}
\cite{1}\cite{2}

Note that a \emph{planar dynamical system} is the same definition as above but with $X$ an open subset of $\mathbb{R}^2$.

\begin{thebibliography}{2}
\bibitem[HSD]{1} Hirsch W. Morris, Smale, Stephen, Devaney L. Robert: Differential Equations, Dynamical Systems \& An Introduction to Chaos \textit{(Second Edition)}.  Elsevier Academic Press, New York, 2004. 
\bibitem[PL]{2} Perko, Lawrence: Differential Equations and Dynamical Systems \textit{(Third Edition)}.  Springer, New York, 2001.
\end{thebibliography}
%%%%%
%%%%%
\end{document}
