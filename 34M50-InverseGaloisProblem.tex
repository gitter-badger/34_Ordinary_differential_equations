\documentclass[12pt]{article}
\usepackage{pmmeta}
\pmcanonicalname{InverseGaloisProblem}
\pmcreated{2013-03-22 15:01:14}
\pmmodified{2013-03-22 15:01:14}
\pmowner{mathcam}{2727}
\pmmodifier{mathcam}{2727}
\pmtitle{inverse Galois problem}
\pmrecord{7}{36728}
\pmprivacy{1}
\pmauthor{mathcam}{2727}
\pmtype{Definition}
\pmcomment{trigger rebuild}
\pmclassification{msc}{34M50}
\pmclassification{msc}{13B05}
\pmclassification{msc}{11R32}
\pmrelated{ShafarevichsTheorem}

\endmetadata

% this is the default PlanetMath preamble.  as your knowledge
% of TeX increases, you will probably want to edit this, but
% it should be fine as is for beginners.

% almost certainly you want these
\usepackage{amssymb}
\usepackage{amsmath}
\usepackage{amsfonts}
\usepackage{amsthm}

% used for TeXing text within eps files
%\usepackage{psfrag}
% need this for including graphics (\includegraphics)
%\usepackage{graphicx}
% for neatly defining theorems and propositions
%\usepackage{amsthm}
% making logically defined graphics
%%%\usepackage{xypic}

% there are many more packages, add them here as you need them

% define commands here

\newcommand{\mc}{\mathcal}
\newcommand{\mb}{\mathbb}
\newcommand{\mf}{\mathfrak}
\newcommand{\ol}{\overline}
\newcommand{\ra}{\rightarrow}
\newcommand{\la}{\leftarrow}
\newcommand{\La}{\Leftarrow}
\newcommand{\Ra}{\Rightarrow}
\newcommand{\nor}{\vartriangleleft}
\newcommand{\Gal}{\text{Gal}}
\newcommand{\GL}{\text{GL}}
\newcommand{\Z}{\mb{Z}}
\newcommand{\R}{\mb{R}}
\newcommand{\Q}{\mb{Q}}
\newcommand{\C}{\mb{C}}
\newcommand{\<}{\langle}
\renewcommand{\>}{\rangle}
\begin{document}
The inverse Galois problem is extremely \PMlinkescapetext{simple} to \PMlinkescapetext{state}, yet \PMlinkescapetext{represents} one of the hardest problems for \PMlinkescapetext{current} \PMlinkescapetext{group} theorists and \PMlinkescapetext{algebraic number theorists}.  It generally asks for descriptions of the \PMlinkescapetext{types} of groups that can occur as Galois groups.  Of course, a significantly more precise formulation is required, for example, because of a result that \PMlinkescapetext{states} that every Galois group is profinite, and every profinite group is a Galois group.  Also ambiguous is what field(s) we allow ourselves to include when computing the Galois group.  Unfortunately, many of these related questions all go under the heading ``the inverse Galois problem,'' so care must be taken to determine an exact formulation of the question being asked.

As an example of a partial solution to this question, it is known that every finite abelian group occurs as the Galois group of an extension over $\mathbb{Q}$ (by the Kronecker-Weber theorem), though it is \emph{not} known whether or not this is true for every finite (not necessarily abelian) group.  This latter question can also be phrased in \PMlinkescapetext{terms} of the absolute Galois group:  ``Does every finite group occur as a quotient group of the absolute Galois group $\text{Gal}(\ol{\mathbb{Q}}/\mathbb{Q})$''?  Thus, an answer to this question would not only reveal information about the nature of finite Galois groups, but also shed light on one of the most elusive objects in all of algebra and number theory.

It is also known (see Shafarevich' theorem) that every solvable group occurs as a Galois group.
%%%%%
%%%%%
\end{document}
