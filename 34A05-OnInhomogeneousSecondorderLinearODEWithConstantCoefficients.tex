\documentclass[12pt]{article}
\usepackage{pmmeta}
\pmcanonicalname{OnInhomogeneousSecondorderLinearODEWithConstantCoefficients}
\pmcreated{2014-03-05 16:25:57}
\pmmodified{2014-03-05 16:25:57}
\pmowner{pahio}{2872}
\pmmodifier{pahio}{2872}
\pmtitle{on inhomogeneous second-order linear ODE with constant coefficients}
\pmrecord{14}{88042}
\pmprivacy{1}
\pmauthor{pahio}{2872}
\pmtype{Derivation}
\pmclassification{msc}{34A05}

% this is the default PlanetMath preamble.  as your knowledge
% of TeX increases, you will probably want to edit this, but
% it should be fine as is for beginners.

% almost certainly you want these
\usepackage{amssymb}
\usepackage{amsmath}
\usepackage{amsfonts}

% need this for including graphics (\includegraphics)
\usepackage{graphicx}
% for neatly defining theorems and propositions
\usepackage{amsthm}

% making logically defined graphics
%\usepackage{xypic}
% used for TeXing text within eps files
%\usepackage{psfrag}

% there are many more packages, add them here as you need them

% define commands here

\begin{document}
Let's consider solving the ordinary second-order linear 
differential equation
\begin{align}
      \frac{d^2y}{dx^2}+a\frac{dy}{dx}+by \;=\; R(x)
\end{align}
which is 
\PMlinkname{inhomogeneous}{HomogeneousLinearDifferentialEquation}, i.e.
$R(x) \not\equiv 0$.

For obtaining the general solution of (1) we have to add to the 
general solution of the 
\PMlinkname{corresponding homogeneous equation}{SecondOrderLinearODEWithConstantCoefficients}
\begin{align}
\frac{d^2y}{dx^2}+a\frac{dy}{dx}+by \;=\; 0
\end{align}
some 
\PMlinkname{particular solution}{SolutionsOfOrdinaryDifferentialEquation} 
of the inhomogeneous equation (1).\, A latter one can 
always be gotten by means of the variation of parameters, but 
in many cases there exist simpler ways to find a particular 
solution of (1).\\

\textbf{$1^\circ$}\!:\; $R(x)$ is a nonzero constant function 
$x \mapsto c$.\, In this case, apparently\, $y = \frac{c}{b}$ is 
a solution of (1), supposing that\, $b \neq 0$.\, If\, $b = 0$\, 
but $a \neq 0$,\, a particular solution is $y = \frac{c}{a}x$.\, 
If\, $a = b = 0$,\, a solution is gotten via two consecutive 
integrations.\\

\textbf{$2^\circ$}\!:\; $R(x)$ is a polynomial function of degree 
$n \ge 1$.\, Now (1) has as solution a polynomial which can be 
found by using indetermined coefficients.\, If\, $b \neq 0$,\, 
the polynomial is of degree $n$ and is uniquely determined.\, 
If\, $b = 0$\, and\, $a \neq 0$,\, the degree of the polynomial 
is $n\!+\!1$ and its constant term is arbitrary. If\, 
$a = b = 0$\, the polynomial is of degree $n\!+\!2$ and is 
gotten via two integrations.\\

\textbf{$3^\circ$}\!:\; Let $R(x)$ in (1) be of the form 
$\alpha\sin{nx}+\beta\cos{nx}$ with $\alpha$, $\beta$, $n$ 
constants.\, We try to find a solution of the same form and put 
into (1) the expression
\begin{align}
y \;:=\; A\sin{nx}+B\cos{nx}.
\end{align}
Then the left hand side of (1) attains the form
$$[(b-n^2)A-anB]\sin{nx}+[anA+(b-n^2)B]\cos{nx}.$$
This must equal $R(x)$, i.e. we have the conditions
$$(b-n^2)A-anB \;=\; \alpha \quad \mbox{and} \quad anA+(b-n^2)B
\;=\; \beta.$$
These determine uniquely the values of $A$ and $B$ provided that
the determinant
$$
\left|\begin{matrix}
b\!-\!n^2 & -an\\
an & b\!-\!n^2
\end{matrix}\right|
\;=\; a^2n^2\!+\!(b\!-\!n^2)^2
$$
does not vanish.\, Then we obtain the particular solution (3).\,   
The determinant vanishes only if\, $a = 0$ and\, $b = n^2$, in 
which case the differential equation (1) reads
\begin{align}
      \frac{d^2y}{dx^2}+n^2y \;=\; \alpha\sin{nx}+\beta\cos{nx}.
\end{align}
Unless we have\, $\alpha = \beta = 0$, the equation (4) has no 
solution of the form (3), since
\begin{align}
\frac{d^2}{dx^2}(A\sin{nx}+B\cos{nx})+n^2(A\sin{nx}+B\cos{nx})
\;=\; 0
\end{align}
identically.\, But we find easily a solution of (4) when we 
differentiate the identity (5) with respect to $n$.\, Changing 
the order of differentiations we get 
$$\frac{d^2}{dx^2}(Ax\cos{nx}-Bx\sin{nx})
+n^2(Ax\cos{nx}-Bx\sin{nx}) \;=\; -2nA\sin{nx}-2nB\cos{nx}.$$
The right hand side coincides with the right hand side of 
(4) iff\; $-2nA = \alpha$\, and\, $-2nB = \beta$, and thus (4) 
has the solution
$$y \;:=\; 
-\frac{\alpha}{2n}x\cos{nx}+\frac{\beta}{2n}x\sin{nx}.$$\\


\textbf{$4^\circ$}\!:\;  Let $R(x)$ in (1) now be 
$\alpha e^{kx}$ where $\alpha$ and $k$ are constants.\, Denote 
the left hand side of (1) briefly
$\frac{d^2y}{dx^2}+a\frac{dy}{dx}+by \;=:\; F(y)$.\, We seek 
again a solution of the same form $Ae^{kx}$ as $R(x)$.

First we have
$$F(Ae^{kx}) \;=\; 
A\underbrace{(k^2\!+\!ak\!+\!b)}_{f(k)}e^{kx} 
\;=\; Af(k)e^{kx}.$$
Thus $A$ can be determined from the condition\, $Af(k) = 
\alpha$.\, If\, $f(k) \neq 0$,\, i.e. $k$ is not a root of the
characteristic equation\, $f(r) = 0$\, corresponding the 
homogeneous equation (2), then we obtain the 
particular solution
$$y \;:=\; \frac{\alpha}{f(k)}e^{kx} 
     \;=\; \frac{\alpha}{k^2\!+\!ak\!+\!b}e^{kx}$$
of the inhomogeneous equation (1).

If\, $f(k) =0$, then $e^{kx}$ and $Ae^{kx}$ satisfy the 
homogeneous equation\, $F(y) = 0$.\, Now we may start from the 
identity
$$F(Ae^{rx}) \;=\; Af(r)e^{rx}$$
and differentiate it with respect to $r$.\, Changing again the 
order of differentiations we can write first
\begin{align}
F(Axe^{rx}) \;=\; Ae^{rx}[f'(r)\!+\!xf(r)],
\end{align}
and differentiating anew,
\begin{align}
F(Ax^2e^{rx}) \;=\; Ae^{rx}[f''(r)\!+\!2xf'(r)\!+\!x^2f(r)].
\end{align}
If $k$ is a simple root of the equation\, $f(r) = 0$,\, i.e. 
if\, $f(k) = 0$\, but\, $f'(k) \neq 0$,\, then\, $r := k$\, 
makes the right hand side of (6) to $Af'(k)e^{kx}$, which 
equals to\, $R(x) = \alpha e^{kx}$\, by choosing\, 
$A := \frac{\alpha}{f'(k)}$.\, Then we have found the 
particular solution
$$y \;:=\; \frac{\alpha}{f'(k)}xe^{kx} 
     \;=\; \frac{\alpha}{2k\!+\!a}xe^{kx}.$$
We have still to handle the case when $k$ is the double root of 
the equation\, $f(k) = 0$\, and thus\, $f'(k) = 0$.\, Putting\,  
$r := k$\, into (7), the right hand side reduces to\, 
$Af''(k)e^{kx} = 2Ae^{kx}$; this equals to\,$R(x) = \alpha e^{kx}$\, 
when choosing $A := \frac{\alpha}{2}$.\, So we have the 
particular solution
$$y \;:=\; \frac{\alpha}{2}x^2e^{kx}$$
of the given inhomogeneous equation.\\


\textbf{$5^\circ$}\!:\; Suppose that in (1) the right hand side 
$R(x)$ is a sum of several functions,
\begin{align}
\frac{d^2y}{dx^2}+a\frac{dy}{dx}+by \;=\; 
R_1(x)+R_2(x)+\ldots+R_n(x),
\end{align}
and one can find a particular solution $y_i(x)$ for each of 
the equations
$$\frac{d^2y}{dx^2}+a\frac{dy}{dx}+by \;=\; R_i(x).$$
Then evidently the sum $y_1(x)+y_2(x)+\ldots+y_n(x)$ is a 
particular solution of the equation (8).\\

\begin{thebibliography}{8}
\bibitem{lindelof}{\sc Ernst Lindel\"of}: {\em Differentiali- ja integralilasku
ja sen sovellutukset III.1}.\, Mercatorin Kirjapaino Osakeyhti\"o, Helsinki (1935).
\end{thebibliography}\\


\end{document}
