\documentclass[12pt]{article}
\usepackage{pmmeta}
\pmcanonicalname{UsingLaplaceTransformToSolveInitialValueProblems}
\pmcreated{2015-05-29 15:21:45}
\pmmodified{2015-05-29 15:21:45}
\pmowner{pahio}{2872}
\pmmodifier{pahio}{2872}
\pmtitle{using Laplace transform to solve initial value problems}
\pmrecord{10}{40922}
\pmprivacy{1}
\pmauthor{pahio}{2872}
\pmtype{Example}
\pmcomment{trigger rebuild}
\pmclassification{msc}{34A12}
\pmclassification{msc}{44A10}
\pmrelated{TableOfLaplaceTransforms}
\pmrelated{LaplaceTransform}

% this is the default PlanetMath preamble.  as your knowledge
% of TeX increases, you will probably want to edit this, but
% it should be fine as is for beginners.

% almost certainly you want these
\usepackage{amssymb}
\usepackage{amsmath}
\usepackage{amsfonts}

% used for TeXing text within eps files
%\usepackage{psfrag}
% need this for including graphics (\includegraphics)
%\usepackage{graphicx}
% for neatly defining theorems and propositions
 \usepackage{amsthm}
% making logically defined graphics
%%%\usepackage{xypic}

% there are many more packages, add them here as you need them

% define commands here

\theoremstyle{definition}
\newtheorem*{thmplain}{Theorem}

\begin{document}
Since the Laplace transforms of the derivatives of $f(t)$ are 
polynomials in the transform parameter $s$ (see table of Laplace 
transforms), forming the Laplace transform of a linear differential 
equation with constant coefficients and initial conditions at\, 
$t = 0$ yields generally a simple equation 
(\PMlinkname{image equation}{imageequation}) for solving the transformed function $F(s)$.\, Since the initial conditions can be taken into consideration instantly, one needs not to determine the general solution of the differential equation.

For example, transforming the equation
$$f''(t)+2f'(t)+f(t) = e^{-t} 
\qquad (f(0) = 0,\;\; f'(0) = 1)$$
gives
$$[s^2F(s)-sf(0)-f'(0)]+2[sF(s)-f(0)]+F(s) = \frac{1}{s+1},$$
i.e. 
$$(s^2+2s+1)F(s) = 1+\frac{1}{s+1},$$
whence
$$F(s) = \frac{1}{(s+1)^2}+\frac{1}{(s+1)^3}.$$
Taking the inverse Laplace transform produces the result
$$f(t) \,=\, te^{-t}+\frac{t^2e^{-t}}{2} \;=\; \frac{e^{-t}}{2}(t^2+2t).$$


%%%%%
%%%%%
\end{document}
