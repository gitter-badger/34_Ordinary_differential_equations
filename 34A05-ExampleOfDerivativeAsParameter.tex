\documentclass[12pt]{article}
\usepackage{pmmeta}
\pmcanonicalname{ExampleOfDerivativeAsParameter}
\pmcreated{2013-03-22 18:29:03}
\pmmodified{2013-03-22 18:29:03}
\pmowner{pahio}{2872}
\pmmodifier{pahio}{2872}
\pmtitle{example of derivative as parameter}
\pmrecord{6}{41160}
\pmprivacy{1}
\pmauthor{pahio}{2872}
\pmtype{Example}
\pmcomment{trigger rebuild}
\pmclassification{msc}{34A05}
\pmsynonym{example of solving an ODE}{ExampleOfDerivativeAsParameter}

% this is the default PlanetMath preamble.  as your knowledge
% of TeX increases, you will probably want to edit this, but
% it should be fine as is for beginners.

% almost certainly you want these
\usepackage{amssymb}
\usepackage{amsmath}
\usepackage{amsfonts}

% used for TeXing text within eps files
%\usepackage{psfrag}
% need this for including graphics (\includegraphics)
%\usepackage{graphicx}
% for neatly defining theorems and propositions
 \usepackage{amsthm}
% making logically defined graphics
%%%\usepackage{xypic}

% there are many more packages, add them here as you need them

% define commands here

\theoremstyle{definition}
\newtheorem*{thmplain}{Theorem}

\begin{document}
For solving the (nonlinear) differential equation 
\begin{align}
x \;=\; \frac{y}{3p}-2py^2
\end{align}
with\, $p = \frac{dy}{dx}$,\, according to \textbf{III} in the \PMlinkname{parent entry}{DerivativeAsParameterForSolvingDifferentialEquations}, we differentiate both sides in regard to $y$, getting first
$$\frac{1}{p} \;=\; \frac{1}{3p}-\left(\frac{y}{3p^2}+2y^2\right)\frac{dp}{dy}-4py.$$
Removing the denominators, we obtain
$$2p +(y+6p^2y^2)\frac{dp}{dy}+12p^3y = 0.$$
The left hand side can be factored:
\begin{align}
(y\frac{dp}{dy}+2p)(1+6p^2y) = 0
\end{align}
Now we may use the zero rule of product; the first factor of the product in (2) yields\, $y\frac{dp}{dy} = -2p$, i.e.
$$2\!\int\frac{dy}{y} = -\!\int\frac{dp}{p}+\ln{C},$$
whence\, $y^2 = \frac{C}{p}$,\, i.e.\, $p = \frac{C}{y^2}$.\, Substituting this into the original equation (1) we get\, $\displaystyle x = \frac{y^3}{3C}-2C$.\, Hence the general solution of (1) may be written
$$y^3 = 3Cx+6C^2.$$
The second factor in (2) yields\, $6p^2y = -1$,\, which is substituted into (1) multiplied by $3p$:
$$3px = y-(-y)$$
Thus we see that\, $p = \frac{2y}{3x}$, which is again set into (1), giving
$$x = \frac{y\cdot3x}{3\cdot2y}-\frac{4y^3}{3x}.$$
Finally, we can write it
$$3x^2 = -8y^3,$$
which (a variant of the so-called semicubical parabola) is the singular solution of (1).


%%%%%
%%%%%
\end{document}
