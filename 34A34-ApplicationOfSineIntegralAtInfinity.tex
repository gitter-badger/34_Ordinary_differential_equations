\documentclass[12pt]{article}
\usepackage{pmmeta}
\pmcanonicalname{ApplicationOfSineIntegralAtInfinity}
\pmcreated{2013-03-22 18:45:58}
\pmmodified{2013-03-22 18:45:58}
\pmowner{pahio}{2872}
\pmmodifier{pahio}{2872}
\pmtitle{application of sine integral at infinity}
\pmrecord{6}{41547}
\pmprivacy{1}
\pmauthor{pahio}{2872}
\pmtype{Application}
\pmcomment{trigger rebuild}
\pmclassification{msc}{34A34}
\pmclassification{msc}{34A12}
\pmclassification{msc}{26A36}
\pmclassification{msc}{26A24}
\pmsynonym{generalisation of sine integral at infinity}{ApplicationOfSineIntegralAtInfinity}
%\pmkeywords{sine integral}

% this is the default PlanetMath preamble.  as your knowledge
% of TeX increases, you will probably want to edit this, but
% it should be fine as is for beginners.

% almost certainly you want these
\usepackage{amssymb}
\usepackage{amsmath}
\usepackage{amsfonts}

% used for TeXing text within eps files
%\usepackage{psfrag}
% need this for including graphics (\includegraphics)
%\usepackage{graphicx}
% for neatly defining theorems and propositions
%\usepackage{amsthm}
% making logically defined graphics
%%%\usepackage{xypic}

% there are many more packages, add them here as you need them

% define commands here
\newcommand{\sijoitus}[2]%
{\operatornamewithlimits{\Big/}_{\!\!\!#1}^{\,#2}}
\begin{document}
For finding the value of the improper integral
\begin{align}
\int_0^\infty\!\frac{\sin{ax}}{x(1\!+\!x^2)}\,dx \;:=\; f(a) \qquad (a > 0) 
\end{align}
we first use the \PMlinkname{partial fraction representation}{PartialFractionsOfExpressions} 
$$\frac{1}{x(1\!+\!x^2)} = \frac{1}{x}-\frac{x}{1\!+\!x^2}.$$
Thus we may write
$$f(a) = \int_0^\infty\frac{\sin{ax}}{x}\,dx-\int_0^\infty\frac{x\sin{ax}}{1+x^2}\,dx.$$
But by the entry sine integral at infinity, the first integral equals $\displaystyle\frac{\pi}{2}$.\, When we check
$$f'(a) \;=\; \int_0^\infty\frac{\cos{ax}}{1\!+\!x^2}\,dx, \quad 
f''(a) \;=\; -\!\int_0^\infty\frac{x\sin{ax}}{1\!+\!x^2}\,dx,$$
we see that there is the linear differential equation
\begin{align}
f(a) = \frac{\pi}{2}+f''(a)
\end{align}
i.e.
$$f''-f \;=\; -\frac{\pi}{2},$$
satisfied by the sought function $a \mapsto f(a)$.\, We have the initial conditions
$$f(0) \;=\; \int_0^\infty{0}\,dx \;=\; 0, \quad f'(0) \;=\; \int_0^\infty\frac{dx}{1\!+\!x^2} 
\;=\; \sijoitus{0}{\quad \infty}\!\arctan{x} \;=\; \frac{\pi}{2}.$$
Therefore the general solution 
$$f(a) \;=\; C_1e^a+C_2e^{-a}+\frac{\pi}{2}$$
of (2) requires that\, $C_1 = 0$,\; $C_2 = \frac{\pi}{2}$,\, and consequently the sought integral $f(a)$ has the value
\begin{align}
\int_0^\infty\!\frac{\sin{ax}}{x(1\!+\!x^2)}\,dx \;=\; \frac{\pi}{2}(1-e^{-a})
\end{align}


%%%%%
%%%%%
\end{document}
