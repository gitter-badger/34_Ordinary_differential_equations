\documentclass[12pt]{article}
\usepackage{pmmeta}
\pmcanonicalname{AllSolutionOfTheLorenzEquationEnterAnEllipsoid}
\pmcreated{2013-03-22 15:15:28}
\pmmodified{2013-03-22 15:15:28}
\pmowner{Daume}{40}
\pmmodifier{Daume}{40}
\pmtitle{all solution of the Lorenz equation enter an ellipsoid}
\pmrecord{4}{37042}
\pmprivacy{1}
\pmauthor{Daume}{40}
\pmtype{Result}
\pmcomment{trigger rebuild}
\pmclassification{msc}{34-00}
\pmclassification{msc}{65P20}
\pmclassification{msc}{65P30}
\pmclassification{msc}{65P40}
\pmclassification{msc}{65P99}

% this is the default PlanetMath preamble.  as your knowledge
% of TeX increases, you will probably want to edit this, but
% it should be fine as is for beginners.

% almost certainly you want these
\usepackage{amssymb}
\usepackage{amsmath}
\usepackage{amsfonts}
\usepackage{amsthm}

% used for TeXing text within eps files
%\usepackage{psfrag}
% need this for including graphics (\includegraphics)
%\usepackage{graphicx}
% making logically defined graphics
%%%\usepackage{xypic} 

% there are many more packages, add them here as you need them

% define commands here

% The below lines should work as the command
% \renewcommand{\bibname}{References}
% without creating havoc when rendering an entry in
% the page-image mode.
\makeatletter
\@ifundefined{bibname}{}{\renewcommand{\bibname}{References}}
\makeatother

\newtheorem{thm}{Theorem}
\newtheorem{defn}{Definition}
\newtheorem{prop}{Proposition}
\newtheorem{lemma}{Lemma}
\newtheorem{cor}{Corollary}
\begin{document}
If $\sigma, \tau, \beta >0$ then all solutions of the Lorenz equation
\begin{eqnarray*}
\dot{x} & = & \sigma(y-x)\\
\dot{y} & = & x(\tau - z) -y\\
\dot{z} & = & xy - \beta z
\end{eqnarray*}
will enter an ellipsoid centered at $(0,0,2\tau )$ in finite time.  
In addition the solution will remain inside the ellipsoid once it 
has entered. To observe this we define a Lyapunov function
$$V(x,y,z)=\tau x^2 + \sigma y^2 + \sigma (z-2\tau )^2.$$  It then
follows that
\begin{eqnarray*}
\dot{V} & = & 2\tau x\dot{x} + 2\sigma y\dot{y} + 2\sigma (z-2\tau )\dot{z}\\
& = & 2\tau x\sigma(y-x) + 2\sigma y(x(\tau - z) -y) + 2\sigma (z-2\tau )(xy - \beta z)\\
& = & -2\sigma (\tau x^2 + y^2 + \beta(z -r)^2 -b\tau^2).
\end{eqnarray*}
We then choose an ellipsoid which all the solutions will enter and 
remain inside.  This is done by choosing a constant $C>0$ such that
the ellipsoid
$$\tau x^2 + y^2 + \beta(z -r)^2 = b\tau^2$$
is strictly contained in the ellipsoid
$$\tau x^2 + \sigma y^2 + \sigma (z-2\tau )^2=C.$$
Therefore all solution will eventually enter and remain inside the above ellipsoid since $\dot{V}<0$ when a solution is located at the exterior of the 
ellipsoid.
%%%%%
%%%%%
\end{document}
