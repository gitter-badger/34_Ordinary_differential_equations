\documentclass[12pt]{article}
\usepackage{pmmeta}
\pmcanonicalname{SecondOrderLinearDifferentialEquationWithConstantCoefficients}
\pmcreated{2013-03-22 13:24:49}
\pmmodified{2013-03-22 13:24:49}
\pmowner{Mathprof}{13753}
\pmmodifier{Mathprof}{13753}
\pmtitle{second order linear differential equation with constant coefficients}
\pmrecord{9}{33960}
\pmprivacy{1}
\pmauthor{Mathprof}{13753}
\pmtype{Topic}
\pmcomment{trigger rebuild}
\pmclassification{msc}{34A30}
\pmclassification{msc}{34-01}
\pmclassification{msc}{34C05}
\pmrelated{GeneralSolutionOfLinearDifferentialEquation}
\pmrelated{TelegraphEquation}
\pmdefines{characteristic equation}
\pmdefines{source}
\pmdefines{sink}
\pmdefines{center}

\endmetadata

% this is the default PlanetMath preamble.  as your knowledge
% of TeX increases, you will probably want to edit this, but
% it should be fine as is for beginners.

% almost certainly you want these
\usepackage{amssymb}
\usepackage{amsmath}
\usepackage{amsfonts}

% used for TeXing text within eps files
%\usepackage{psfrag}
% need this for including graphics (\includegraphics)
%\usepackage{graphicx}
% for neatly defining theorems and propositions
%\usepackage{amsthm}
% making logically defined graphics
%%%\usepackage{xypic}

% there are many more packages, add them here as you need them

% define commands here
\begin{document}
Consider the second order homogeneous linear differential equation
\begin{equation}
x''+bx'+cx=0,
\label{eq}
\end{equation}
where $b$ and $c$ are real constants. 

The explicit solution is easily found using the characteristic equation method. This method, introduced by Euler, consists in seeking solutions of the form $x(t)=e^{rt}$ for (\ref{eq}). Assuming a solution of this form, and substituting it into (\ref{eq}) gives
\[
r^2 e^{rt}+bre^{rt}+ce^{rt}=0.
\]
Thus
\begin{equation}
r^2+br+c=0
\label{char_eq}
\end{equation}
which is called the \emph{characteristic equation} of (\ref{eq}). Depending on the nature of the \PMlinkname{roots}{Equation} $r_1$ and $r_2$ of (\ref{char_eq}), there are three cases.
\begin{itemize}
\item If the roots are real and distinct, then two linearly independent solutions of (\ref{eq}) are
\[
x_1(t)=e^{r_1t},\quad x_2(t)=e^{r_2t}.
\]
\item If the roots are real and equal, then two linearly independent solutions of (\ref{eq}) are
\[
x_1(t)=e^{r_1t},\quad x_2(t)=te^{r_1t}.
\]
\item If the roots are complex conjugates of the form $r_{1,2}=\alpha\pm i\beta$, then two linearly independent solutions of (\ref{eq}) are
\[
x_1(t)=e^{\alpha t}\cos \beta t,\quad x_2(t)=e^{\alpha t}\sin \beta t.
\]
\end{itemize}
The general solution to (\ref{eq}) is then constructed from these linearly independent solutions, as
\begin{equation}
\phi(t)=C_1x_1(t)+C_2x_2(t).
\label{sol}
\end{equation}

Characterizing the behavior of (\ref{sol}) can be accomplished by studying the two-dimensional linear system obtained from (\ref{eq}) by defining $y=x'$:
\begin{align}
x'&=y \\
y'&=-by-cx.
\label{sys}
\end{align}
Remark that the roots of (\ref{char_eq}) are the eigenvalues of the Jacobian matrix of (\ref{sys}). This generalizes to the characteristic equation of a differential equation of order $n$ and the $n$-dimensional system associated to it. 

Also note that the only equilibrium of (\ref{sys}) is the origin $(0,0)$. Suppose that $c\neq 0$. Then $(0,0)$ is called a
\begin[roman]{enumerate}
\item \emph{source} iff $b<0$ and $c>0$,
\item \emph{spiral source} iff it is a source and $b^2-4c<0$,
\item \emph{sink} iff $b>0$ and $c>0$,
\item \emph{spiral sink} iff it is a sink and $b^2-4c<0$,
\item \emph{\PMlinkescapetext{saddle}} iff $c<0$,
\item \emph{center} iff $b=0$ and $c>0$.
\end{enumerate}
%%%%%
%%%%%
\end{document}
