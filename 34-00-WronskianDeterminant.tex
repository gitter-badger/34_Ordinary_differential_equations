\documentclass[12pt]{article}
\usepackage{pmmeta}
\pmcanonicalname{WronskianDeterminant}
\pmcreated{2013-03-22 12:22:59}
\pmmodified{2013-03-22 12:22:59}
\pmowner{rspuzio}{6075}
\pmmodifier{rspuzio}{6075}
\pmtitle{Wronskian determinant}
\pmrecord{13}{32164}
\pmprivacy{1}
\pmauthor{rspuzio}{6075}
\pmtype{Definition}
\pmcomment{trigger rebuild}
\pmclassification{msc}{34-00}
\pmsynonym{Wronskian}{WronskianDeterminant}
\pmrelated{GrammianDeterminant}

% this is the default PlanetMath preamble.  as your knowledge
% of TeX increases, you will probably want to edit this, but
% it should be fine as is for beginners.

% almost certainly you want these
\usepackage{amssymb}
\usepackage{amsmath}
\usepackage{amsfonts}

% used for TeXing text within eps files
%\usepackage{psfrag}
% need this for including graphics (\includegraphics)
%\usepackage{graphicx}
% for neatly defining theorems and propositions
%\usepackage{amsthm}
% making logically defined graphics
%%%\usepackage{xypic} 

% there are many more packages, add them here as you need them

% define commands here
\begin{document}
Given functions $f_1, f_2, \dotsc, f_n$, then the \emph{Wronskian determinant} (or simply the Wronskian) $W(f_1, f_2, f_3, \dotsc, f_n)$ is the determinant of the square matrix
\[
W(f_1, f_2, f_3, \dotsc, f_n) = \left\lvert\begin{array}{@{}ccccc@{}}
f_1 & f_2 & f_3 & \cdots & f_n\\
f_1' & f_2' & f_3' & \cdots & f_n'\\
f_1'' & f_2'' & f_3'' & \cdots & f_n''\\
\vdots & \vdots & \vdots & \ddots & \vdots\\
f_1^{(n-1)} & f_2^{(n-1)} & f_3^{(n-1)} & \cdots & f_n^{(n-1)}\\
\end{array}\right\rvert
\]
where $f^{(k)}$ indicates the $k$th derivative of $f$ (not exponentiation).

The Wronskian of a set of functions $F$ is another function, which is zero over any interval where $F$ is linearly dependent. Just as a set of vectors is said to be linearly dependent when there exists a non-trivial linear relation between them, a set of functions $\{f_1, f_2, f_3, \dotsc, f_n\}$ is also said to be dependent over an interval $I$ when there exists a non-trivial linear relation between them, i.e.,
\[
a_1 f_1(t) + a_2 f_2(t) + \dotsb + a_n f_n(t) = 0
\]
for some $a_1, a_2, \dotsc, a_n$, not all zero, at any $t \in I$.

Therefore the Wronskian can be used to determine if functions are independent. This is useful in many situations. For example, if we wish to determine if two solutions of a second-order differential equation are independent, we may use the Wronskian.

\paragraph{Examples}

Consider the functions $x^2$, $x$, and $1$. Take the Wronskian:
\[
W = \left\lvert\begin{array}{@{}ccc@{}}
x^2 & x & 1\\
2x & 1 & 0\\
2 & 0 & 0\\
\end{array}\right\rvert = -2
\]
Note that $W$ is always non-zero, so these functions are independent everywhere. Consider, however, $x^2$ and $x$:
\[
W = \left\lvert\begin{array}{@{}cc@{}}
x^2 & x\\
2x & 1\\
\end{array}\right\rvert = x^2 - 2x^2 = -x^2
\]
Here $W = 0$ only when $x = 0$. Therefore $x^2$ and $x$ are independent except at $x = 0$.

Consider $2x^2+3$, $x^2$, and $1$:
\[
W = \left\lvert\begin{array}{@{}ccc@{}}
2x^2 + 3 & x^2 & 1\\
4x & 2x & 0\\
4 & 2 & 0\\
\end{array}\right\rvert = 8x - 8x = 0
\]
Here $W$ is always zero, so these functions are always dependent. This is intuitively obvious, of course, since
\[
2x^2 + 3 = 2(x^2) + 3(1)
\]

Given $n$ linearly independant functions $f_1, f_2, \dotsc, f_n$, we can use 
the Wronskian to construct a linear differential equation whose solution space
is exactly the span of these functions.  Namely, if $g$ satisfies the equation
\[
W(f_1, f_2, f_3, \dotsc, f_n, g) = 0,
\]
then $g = a_1 f_1(t) + a_2 f_2(t) + \dotsb + a_n f_n(t)$ for some choice of
$a_1, a_2, \dotsc, a_n$.

As a simple illustration of this, let us consider polynomials of at most second order.  Such a polynomial is a linear combination of $1$, $x$, and $x^2$.  We
have 
\[ W (1, x, x^2, g(x)) = \left| \begin{matrix} 1 & x & x^2 & g(x) \\
0 & 1 & 2x & g'(x) \\ 0 & 0 & 2 & g''(x) \\ 0 & 0 & 0 & g'''(x)
\end{matrix} \right| = 2 g''' (x) .\]
Hence, the equation is $g''' (x) = 0$ which indeed has exactly polynomials
of degree at most two as solutions.
%%%%%
%%%%%
\end{document}
