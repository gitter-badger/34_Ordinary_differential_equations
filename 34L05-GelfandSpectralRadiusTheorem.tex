\documentclass[12pt]{article}
\usepackage{pmmeta}
\pmcanonicalname{GelfandSpectralRadiusTheorem}
\pmcreated{2013-03-22 13:39:19}
\pmmodified{2013-03-22 13:39:19}
\pmowner{Andrea Ambrosio}{7332}
\pmmodifier{Andrea Ambrosio}{7332}
\pmtitle{Gelfand spectral radius theorem}
\pmrecord{9}{34309}
\pmprivacy{1}
\pmauthor{Andrea Ambrosio}{7332}
\pmtype{Theorem}
\pmcomment{trigger rebuild}
\pmclassification{msc}{34L05}
\pmsynonym{spectral radius formula}{GelfandSpectralRadiusTheorem}
\pmrelated{SelfConsistentMatrixNorm}

\endmetadata

% this is the default PlanetMath preamble.  as your knowledge
% of TeX increases, you will probably want to edit this, but
% it should be fine as is for beginners.

% almost certainly you want these
\usepackage{amssymb}
\usepackage{amsmath}
\usepackage{amsfonts}
\newcommand{\mv}[1]{\mathbf{#1}}

% used for TeXing text within eps files
%\usepackage{psfrag}
% need this for including graphics (\includegraphics)
%\usepackage{graphicx}
% for neatly defining theorems and propositions
%\usepackage{amsthm}
% making logically defined graphics
%%%\usepackage{xypic}

% there are many more packages, add them here as you need them

% define commands here
\begin{document}
For every self-consistent matrix norm, $||\cdot||$, and every square matrix $\mv{A}$ we can write\\
\begin{displaymath}
\rho(\mv{A})=\lim_{n \to \infty} ||\mv{A}^n||^{\frac{1}{n}}.
\end{displaymath}\\

Note: $\rho(\mv{A})$ denotes the spectral radius of $\mv{A}$.

This theorem also generalizes to infinite dimensions and plays an important role in the theory of operator algebras.  If $\mathcal{A}$ is a Banach algebra with norm $||\cdot||$ and $A \in \mathcal{A}$, then we have\\
\begin{displaymath}
\rho(\mv{A})=\lim_{n \to \infty} ||\mv{A}^n||^{\frac{1}{n}}.
\end{displaymath}\\

It is worth pointing out that the self-consistency condition which was imposed on the matrix norm is part of the definition of a Banach algebra.  A common case of the infinite-dimensional generalization  occurs when $\mathcal{A}$ is the algebra of bounded operators on a Hilbert space --- the operators may be regarded as an infinite-dimensional generalization of the square matrices.
%%%%%
%%%%%
\end{document}
